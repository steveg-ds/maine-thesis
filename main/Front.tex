\copyrightpage[Steve Garcia]{2024}		% Optional, comment out or delete if undesired

\begin{abstract}
This thesis explores rural housing insecurity through Swope and Hernandez’s (2019) 4 C's of housing insecurity in rural areas. Little attention has been paid to rural areas in the conversation on housing insecurity and houselessness \citep{gleason_using_2021}. To facilitate further discussion on this understudied issue, this exploratory study used unsupervised machine learning to group census tracts into risk levels across seven axes of data from the American Community Survey. The axes were based on housing insecurity factors found in the literature. \textit{K}-medoids clustering is used to group census tracts into high, medium, and low risk of housing insecurity for each axes. Multinomial logistic regression was used to determine variation between U.S. states based on how well state risk levels could be predicted with the national dataset. Furthermore, spatial autocorrelation analysis was employed to gauge the extent of spatial clustering within the identified risk levels and housing insecurity factors. The results indicate that many rural census tracts have a medium risk of housing insecurity, and the risk levels are hard to predict. The spatial autocorrelation results show that the housing insecurity variables were not as highly spatially clustered as expected.  
\end{abstract}

\begin{layabstract}{Housing Insecurity, homelessness, data mining}	% Replace the ... with the list of keywords
This thesis delves into the challenges of insecure housing in rural areas, drawing from Swope and Hernandez's comprehensive 4 C's framework outlined in 2019. Rural regions have been overlooked in conversations about housing insecurity and homelessness \citep{gleason_using_2021}. To shed light on this neglected issue, this investigation utilized unsupervised machine learning techniques. It categorized census tracts into risk levels across seven key data axes sourced from the American Community Survey, focusing on factors commonly associated with housing insecurity as documented in existing literature. By employing multinomial logistic regression, the study aimed to examine the predictability of housing insecurity risk among U.S. states. Furthermore, spatial autocorrelation analysis was employed to gauge the extent of spatial clustering within the identified risk levels and housing insecurity factors. The findings uncover that numerous rural census tracts exhibit a moderate risk of housing insecurity, yet predicting these risk levels proves challenging. Intriguingly, the spatial autocorrelation analysis suggests that the housing insecurity variables didn't exhibit the anticipated high levels of spatial clustering.


\end{layabstract}

\begin{dedication}
Dedicated to St. Thomas More and all who seek to build a better world.
\end{dedication}

\begin{acknowledgements}
I extend my heartfelt appreciation to the following individuals for their invaluable contributions to this research project:
    
    My advisor, Dr. Matthew Dube, who provided unwavering guidance and feedback throughout this endeavor; my committee members, Drs. Kristen Gelason and Sarah Walton for their insightful feedback and support; my beloved wife, Katie, whose encouragement has been a constant source of motivation; Drs. Megan Smith, Jeremy Mhire, and Dolliann Hurtig for their mentorship and guidance during my academic journey; and my grandparents, Steve and Linda Mauldin, for their wisdom and support throughout life's challenges.
    
    Your collective support has been instrumental in the success of this project, and I am sincerely grateful for your presence in my life."
\end{acknowledgements}

\pagebreak

% Commands for the required lists
\tableofcontents

\pagebreak

\listoftables	

% Include only if there are tables in the thesis
\pagebreak
\listoffigures		
\pagebreak		% Include only if there are figures in the thesis


%\listof{equations}{List of Equations}

% Sets the document spacing and pagestyle.
\mainmatter

\endinput
