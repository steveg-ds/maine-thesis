\chapter{Introduction} 

%\subsection{\textit{Background}}
Homelessness research has undergone a significant transformation in recent years. Traditionally, the focus was on categorizing and describing different segments of the homeless population \citep{lee_homelessness_2021}. A more contemporary approach views homelessness as a spectrum rather than a binary condition (e.g., Desmond et al., 2015; Swope and Hernandez, 2019). This paradigm shift opens opportunities for preventative and reactive services to address homelessness and \hs. To harness these opportunities, four critical areas in literature require attention. First, housing research has concentrated on urban settings resulting in an urban-centric view of social issues like poverty and homelessness. Second, measuring \hs proves challenging due to its dependence on circumstances and obstacles for individuals and communities \citep{leifheit_building_2022}. Third, housing and homelessness in urban and rural areas necessitates a multi-disciplinary approach to properly capture the aspects that contribute to them, an approach that has been scarcely taken in the extant literature.  Finally, the scarcity of identified community-level risk factors in rural areas, coupled with a dearth of rural-specific data and research, makes the study of rural homelessness particularly intricate \citep{gleason_using_2021}. Addressing these gaps by integrating rural areas into the discourse on homelessness and \hs is essential for creating a just and equitable society with effective policies for preventing and addressing homelessness\citep{oregan_how_2021}.

\section{\textit{Rural Areas}}
Rural areas differ vastly across regions and even states, often used as a catch-all for non-urban areas. Rurality is often defined simply as not being urban \hl{Robertson et al., 2007)}. Defining rural areas in contrast to urban areas largely excludes the variation between rural areas. The lack of universally accepted definitions of rurality reduces the amount of time and resources that can be dedicated to struggling communities \hl{Samadura and Yousey 2018}. Rural areas encompass a broad spectrum, including farms, ranches, villages, small towns, and many other characteristics \citep{cromartie_defining_2008}. At their core, rural areas are a function of "space, distance, and relative population density" \citep[?]{castle_place_2011}.\citet{shoup_principles_2010} group urban areas into three categories: rural ares dependent on nearby urban centers, "destination counties" with natural or artificial amenities that attract non-permanent residents, and production communities that revolve primarily around a single industry. This wide variation between rural areas makes defining and understanding rurality a difficult challenge. Rural areas dominate the land mass of the United States, but with 85 percent of the population living in urban areas, they are often overlooked in the public discussion \hl{(Pendall et al. 2016}) In the study of housing, rural areas are almost entirely excluded from the conversation \citep{gkartzios_housing_2017}. Contributing significantly to this problem is a wide variety of definitions used by governmental organizations, policy makers, and scholars (\hl{Samadura and Yousey, 2018}; \citealp{cromartie_defining_2008}). Recently, The main policy objective for rural communities has been the promotion of economic development and preservation of the characteristics ascribed to rural areas \citep{lichter_changing_2007}. 

\textit{\hl{Figure 1.1: Map of Rural areas here vs population density here}} % use ACS data to make this map 

As Figure 1.1 demonstrates, the size of rural areas compared to the population density is vast and deconstructing the urban-centric lens of housing research necessitates a novel approach that can accommodate the differences in rural areas. The size and variation necessitates addressing rural issues differently because there can be no one size fits all policy approach to improving conditions. 

\section{\textit{Literal Homelessness}}
For decades, scholars have debated if research should focus on the reasons why people become homeless or the structural forces that create homelessness \citep{shlay_social_2003}. Prior to significant shifts in the 21st century, research on homelessness focused on identifying and describing categories of homeless people \citep{lee_homelessness_2021}. In other words, much research has focused on the binary of individuals and families being housed or unhoused and trying to assign them into umbrella categories. This neglects the wide range of individual and societal factors that occur in the phases between when an individual or family is housed and becomes unhoused. Estimating the number of people that are unhoused is notoriously difficult even in urban areas. For measuring homelessness, the most popular mechanism in the United States is the Department of Housing and Urban Development (HUD) point-in-time (PIT) count and housing inventory count. These counts are used for the distribution of federal funds for combating homelessness. as \hl{Agans et al. (2014)} note that the unhoused frequently relocate and the housed may quickly become unhoused, making it difficult to accurately estimate the number of unhoused people at any given time. When it comes to addressing literal homelessness, public health experts differentiate between preventative services and reactive or emergency services \citep{oregan_how_2021}. Preventive services prevent households from becoming homeless, while reactive or emergency services step in after a household becomes homeless. As homelessness is often seen as an urban problem, most intervention occurs in urban areas \citep{gleason_using_2021}. 

\section{\textit{Housing as Health}}
A house is far more than four walls, a roof, and some doors. The characteristics and location of a house can make a significant impact on one’s life. In the United States, housing is often a family’s greatest expenditure, greatest source of wealth, and a place of safety and gathering \citep{braveman_housing_2011}. The federal government has long acknowledged this through legislation like the Housing Act of 1949, and social programs and development goals developed by the U.S. Department of Housing and Urban Development. Housing is often seen as one of the most fundamental determinants of health, and a lack of adequate housing can produce adverse health outcomes and acts as a foundation for “social, psychological, and cultural well-being” (\citealp[p.17]{dalessandro_housing_2020}; \citealp{leifheit_building_2022}). Part of acknowledging housing as health is moving beyond the housed/ unhoused binary in order to better understand and intervene in households that are at risk of being unhoused. This is often referred to as \hs, a broader term that encompasses a continuum that affects a larger part of the population than simply housed/ unhoused \citep{deluca_housing_2022}.

\section{\textit{Theoretical Framework}}
With little infrastructure for homelessness services in rural areas, the 4 C’s approach to \hs proposed by \citet{swope_housing_2020} can highlight areas of critical concern for devoting resources to reactive services and identify areas where preventative services can improve or expand. The pillars of the 4 C's include:
\begin{itemize}
    \item{Conditions: quality of housing}
    \item{Cost: housing affordability}
    \item{Consistency: residential stability}
    \item{Context: neighborhood opportunity }
\end{itemize}
The 4 C’s housing as an interconnected web of factors that impact health and encapsulates “this unequal distribution of housing disparities along other axes of inequality, and the historical forces shaping unequal housing opportunities” \citep[1]{hernandez_housing_2019}. Swope and Hernandez are not the only scholars to design a model encompassing these 4 factors. \citet{metzger_fair_2017} proposed a similar framework that encompasses stability, affordability, internal housing conditions, and area characteristics. 



\section{\textit{Motivation}}
There are three primary reasons behind this exploration:
\begin{itemize}
    \item{The first motivation comes from the lack of attention scholars have paid to rural areas as it pertains to \hs. While the literature on rural housing insecurity is growing, there has yet to be a holistic nationwide survey of rural housing insecurity. Rural areas deserve more attention, and this thesis hopes to serve as a starting point for future research on rural housing insecurity at all levels with the ultimate goal of breaking the urban-focused lens of housing insecurity.}
    \item{The second motivation is to test the efficacy of the 4 C's model of \hs for the rural United States. As there is an urban lens to \hs, an adequate theoretical model must be capable of adapting to areas often left out of the conversation. One study \citep{gleason_using_2021} has applied the 4 C's model to housing insecurity for the state of Maine. Most applications of the 4 C's has been to study the relationships between various conditions and housing insecurity, but no studies have applied it broadly to rural housing insecurity.}
    \item{The final motivation is to provide policy-makers and researchers with a quantitative and qualitative framework to identify rural areas of \hs in their constituency and create harm reduction approaches and services that can meet the unique needs of their areas. The patchwork of local, state, and federal systems that encompass the aid programs of the United States means that there are a lot of people involved in the policy-making process with, as of yet, no real mechanism for addressing housing insecurity in their particular jurisdictions.}

\end{itemize}

\section{\textit{Approach}}
In order to improve our understanding of rural \hs, this thesis investigates the risk levels of rural census tracts in the United States  using the 4 C's model of \hs. Using k-medoid clustering, risk factors across eight different sectors are used to assign housing insecurity risk levels to rural census tracts for the continental United States. Each state is clustered with census tracts from other states within a 15-mile boundary to encapsulate how communities exist across state lines. The cluster medians are analyzed to understand the trends in \hs factors across states and the clusters are relabeled based on risk factors identified in the literature so that each cluster falls into a low, medium, or high risk level. Using these risk levels, association rules learning is used to identify common patterns between sector risk levels and identify pockets of rural \ct that are at high risk of \hs. To better understand how factors of \hs relate to space, Moran's I spatial autocorrelation is used to determine how spatially clustered each \hs factor is. Local Moran's I is used to determine how spatially clustered each risk level is to better understand the clustering of housing insecurity risk in rural areas.Finally, a multinomial logistic regression is used to determine how well each states sector risk levels can be predicted and a national model is generated for each sectors risk levels to analyze well the risk levels created by this implementation of the 4 C's model can be predicted nationally and state by state. 

Beginning to understand rural homelessness requires that several questions be answered:
\begin{itemize}
    \item How can risk factors of be used to identify risk levels of \hs while accounting for the variation in rural areas? 
    \item Do \hs factors identified for urban areas exhibit similar characteristics in rural areas? 
    \item When measuring \hs across different dimensions, how often do the same features arise?
    \item Are there spatial relations between the different dimensions of \hs? 
    \item To what extent can this model of housing insecurity be used to predict risk levels across housing insecurity factors?

\end{itemize}
\section{\textit{Major Results}}
I have some

\section{\textit{Intended Audience}}
This thesis is intended for an audience with a significant interest in rural housing insecurity. Such an audience can include, but is not limited to policy makers, economists, political scientists, community psychologists, rural sociology, and many others concerned with housing insecurity. 

\section{\textit{Structure of Thesis}}
The thesis is structured into six chapters. Chapter 2 offers a comprehensive theoretical foundation, focusing on the application of the 4 C's of the \hs model. This chapter critically reviews pertinent literature pertaining to various facets of the model. Chapter 3 delineates the methodology employed for data processing. It provides an in-depth explanation of the chosen methodology and its execution. The ensuing Chapter 4 presents the study's findings, offering a detailed analysis of the acquired results. Chapter 5 deliberates on the implications of the results, discussing their significance and impact within the scope of the study. This chapter provides a thorough examination of the noteworthy findings. Chapter 6 serves as a synthesis, summarizing the entirety of the work and offering insightful commentary on the major findings. Additionally, it delineates potential avenues for future research and study.

\endinput