\chapter{Discussion and Future Directions}	

Housing insecurity affects all aspects of life for the individuals that experience it, and it has grave consequences for communities. This is especially true for rural areas where a conundrum of factors over recent decades has reduced the amount of community resources available for combatting housing insecurity and homelessness. These problems include a lack of uniform definitions, persistent poverty, and hardships from economic changes (\citealp{yousey_defining_2018}; \citealp{crandall_local_2004}; \citealp{pendall_future_2016}; \citealp{kropczynski_insights_2012}). Unfortunately, our understanding of how these factors affect housing insecurity and homelessness in rural areas is limited due to the urban-centric lens used by researchers and policymakers. The present analyses examined variables associated with housing insecurity in a sample of rural census tracts based on RUCA designations to group census tracts into risk levels that show similar signs of housing insecurity risk. Several data mining techniques were then applied to analyze how risk levels and variables relate to each other. The results show a notable amount of census tracts at a high or medium risk of housing insecurity and provide evidence that there is great variation in the housing insecurity risk in rural areas at the census tract level. 

\section{\textit{Identifying and Analyzing Risk Levels}}

Risk factors of housing insecurity were used to identify levels of housing insecurity risk while accounting for the variation in rural areas with a combination of unsupervised machine learning and a spatial neighbors algorithm. \textit{k}-medoid clustering was used to cluster census tracts with similar values across eight different sectors of variables, encapsulating different aspects of housing insecurity. To account for the variation in rural areas, a neighbors algorithm was used so that bordering census tracts that make up rural communities could be included in the clustering for each state. There are three benefits to this approach. First, clustering by each state and neighboring census tracts prevents the most vulnerable communities in one state from influencing the risk level of the most vulnerable communities in another state. Second, this state-by-state approach makes this research a tool that policymakers and researchers can use in their states. Finally, and most significantly, a relative approach to measuring housing insecurity can capture the variation in rural areas better than an absolute approach with rigidly defined categories. Whether or not this methodology adequately allows for these differences requires further research to validate.

Previous research has identified both pockets of "prosperous" rural areas and rural areas that are considered pockets of poverty (\citealp{isserman_why_2009}; \citealp{miller_persistent_2003}). A similar process is used whereby a threshold was established to differentiate between high, medium, and low-risk census tracts in terms of housing insecurity across sectors using supervised machine learning. The cluster analysis highlighted 280 census tracts identified as high-risk and 1,692 census tracts identified as medium-risk (see Figure~\ref{fig:regional_risk_map}). These are areas of concern as they all have an increased amount of high and medium risk levels across sectors relative to other census tracts in their state. These census tracts show consistent signs of housing insecurity relative to other census tracts in their state. 

 The cluster analysis shows several important observations about rural areas. First is the importance of education, health, and social work employment in rural areas (See Table~\ref{tab:emp}). The cluster medians for each risk level is nine percent. This is significant because the variables for jobs that rural areas typically depend on like agriculture, construction, and forestry, have notably lower averages and cluster medians. The impact of this is two-fold. First, many of the jobs that fall into education, health, and social work fall into employment in the public sphere, which is affected by the decreases in funding caused by several processes affecting rural areas \citep{blank_poverty_2005}. Second, it demonstrates the decline of manufacturing and agriculture in rural areas. These industries used to dominate rural areas but scholars have identified a significant decline in their prevalence over several decades \citep{robertson_rural_2007}. Public administration is the most stable sector across clusters with values remaining almost identical across risk levels. One unexpected result is the significant presence of retail trade, which has higher cluster medians than most of the job sectors considered "rural". The clustering results provide further evidence of the shift away from traditional rural economies that scholars have identified (\citealp{pendall_future_2016}; \citealp{blank_poverty_2005}).
 
 The demographic variable cluster results align with previous studies on the presence of pockets of minorities in rural areas. The cluster with the lowest white percentage of the population has the highest percentage of Black, Hispanic or Latino, and Other race variables. Concentrations of minorities are espectially noteworthy because they are subjected to a variety of historical processes that put them at a higher risk of housing insecurity. Particularly concerning are concentrations of African Americans in the South, where the effects of segregation are still seen today. The presence of Hispanics and Latinos in the same cluster as the lowest white population and the highest median black population is interesting because the migration of Hispanics and Latinos to rural areas has been indicated as a potential solution to the well-documented population issues facing rural areas \citep{lichter_demographic_2020}. The clustering results also provide further evidence of the aging population of rural areas, with the number of males and females over 18 being almost three times higher than the number of males and females under 18 across all three risk levels. This reflects the trend of rural areas aging faster than urban areas \citep{cohen_aging_2022}. 
 
For housing costs, the cluster medians indicate that while home ownership is widespread in rural areas, there are generally more high-cost renters than owners. The cluster medians for high-cost renters range from 14 to 17 percent while homeowners with a mortgage and high housing costs range from 4 to 5 percent. This indicates that renters in rural areas may face similar issues as those in urban areas when it comes to the affordability of rental property. High housing costs can create a vicious cycle where low-income households must move frequently, often to worse neighborhoods with decreasing housing conditions \citep{desmond_forced_2015}. Segregation in rural areas is a concern for homeownership as well, \citet{krivo_housing_2004} found that Hispanics and African Americans face discrimination in the housing market that suppresses the accumulation of home equity. It must be remembered that the ability to access affordable housing is largely determined by demographic characteristics which in turn are influenced by historical forces such as discrimination (\citealp{yadavalli_comprehensive_2020}, \citealp{hernandez_housing_2019}). 

The housing quality cluster analysis does not show any concerning results. Between 17 and 25 percent of unoccupied housing have incomplete kitchens or incomplete plumbing based on the cluster medians. While it is expected that some housing will not be habitable due to a lack of maintenance, it is unclear if these numbers are high or to be expected. Occupied houses with these problems are much less common with all cluster medians less than one percent.  
 
The RME sector shows relatively low levels of transiency among those who have either a high school diploma or did not finish high school. Researchers should be concerned with the cluster medians between 7 to 8 percent of the population that are stable but do not have a high school education. The value of a high school diploma is well understood, so areas with low levels of high school graduation require further attention. Rural areas are particularly vulnerable to these problems in education in areas facing population loss and economic problems because they are left with a lower tax base and less funding for schools, potentially increasing the likelihood of higher dropout rates \citep{blank_poverty_2005}.
 
The most notable observation from the RMP cluster analysis is that the number of households that live below the poverty level is higher than those slightly above it across all three risk levels. This is concerning due to the well-documented impacts of poverty. At the extreme end, severe poverty can lead to literal homelessness and households experiencing poverty often face housing insecurity (\citealp{evans_reducing_2019}; \citealp{cox_road_2019}). 
 
 The cluster analysis shows three points of concern in rural areas for the household factors section. First, for households with no investment income and no other income, all cluster medians are above 30 percent. This indicates that a significant number of households in rural areas are not building wealth through means outside of wages received from employment. This is concerning given the rise of economic insecurity \citep{desmond_housing_2016-1}. Second, a median of five percent of the population receives public assistance across all risk levels. This reflects that rural areas may not be fully taking advantage of assistance that could improve their living conditions. Previous research has documented the tendency for people to not use public assistance for various reasons \citep{lichter_changing_2007}. Third, the median Gini index for each risk level is between 0.42 and 0.44, whereas the national Gini index is 0.488 in 2022 \citep{us_census_bureau_income_2023}. Although income inequality is often considered an urban problem, there is only slightly less income inequality in rural areas compared to the nation.

The cluster analysis for housing type shows that single-unit renters and owners are the predominant means of housing in rural areas. One area of concern not accounted for here is the presence of mobile home parks. Research has shown that practices allowed in mobile home parks can put some at a higher risk of housing insecurity \citep{mactavish_policy_2006}. For the high-risk housing type census tracts, 10 percent of owners and renters each live in mobile homes based on the cluster medians. These areas where there are higher rates of renters relative to owners than other areas warrant further attention because there may be some unidentified factors in the community that contribute to the lower amounts of home ownership. 

The cluster analysis of rural areas reveals several significant findings. Education, health, and social work employment emerge as vital sectors, contrasting with declining traditional rural industries like agriculture and manufacturing. Demographically, clusters with high minority populations highlight historical processes impacting housing insecurity, with concentrations of African Americans and Hispanics in particular areas. Housing affordability challenges persist, with high-cost renters more prevalent than owners, reflecting urban-like housing dynamics. Maintenance issues in unoccupied housing suggest ongoing challenges in housing quality. Additionally, low high school graduation rates and high poverty levels underscore economic insecurity in rural communities. Limited wealth-building opportunities and relatively low public assistance usage further contribute to rural economic vulnerability. The presence of mobile homes presents unique challenges, especially in high-risk areas. Overall, these findings underscore the complexity of rural housing dynamics, requiring multifaceted solutions to address economic, demographic, and housing quality challenges.

\section{\textit{Patterns of Risk}}

Under the 4 Cs model, there is an implicit assumption that areas at a high level of risk in one sector may have a higher level of risk in another sector because the pillars are an interconnected web. The association rules analysis reveals several interesting findings about the co-occurrence of housing insecurity risk. First, An area at an overall high risk of housing insecurity would at least have a high-risk level across more than one sector. Tables~\ref{tab:high_risk_ass},~\ref{tab:low_risk_ass},~\ref{tab:low_high_risk},~\ref{tab:high_low_risk} show the frequency of the most general trends of the association rules: high-risk-to-high-risk, low-risk-to-low-risk, low-risk-to-high-risk, and high-risk-to-high-risk. The most surprising result from this analysis is similar levels of presence between these relationships. The trends in association rules indicate that few areas exhibit a risk of housing insecurity across multiple sectors. This is reflected by Figure~\ref{fig:regional_map} where the map of housing insecurity shows some pockets of red, but vast amounts of green and yellow indicating a low and medium risk of housing insecurity across sectors for most rural areas. The lack of a significant number of rules with more than one element on the left-hand side provides further evidence for this hypothesis. This raises further questions related to the clustering of housing insecurity factors among rural areas as is often seen in urban areas. The most interesting finding from the risk analysis is the high presence of high-risk census tracts in the most rural areas. 40 to 50 percent of high-risk census tracts are classified as the most rural across all sectors. Second to the most rural areas are small towns, with 30 to 40 percent of high-risk census tracts across all sectors. Only four to 15 percent of high-risk census tracts are classified as an eight or a nine on the Rural-Urban Commuting Area Codes scale. These results show that the most rural areas are at the highest risk of housing insecurity, next to the most urban areas considered in this dataset. 

As the goal of this research is not just to provide an exploratory analysis of rural housing insecurity but also to serve as a tool for further study, it's important to highlight areas that have a high risk of housing insecurity. The first way this was accomplished was through the risk threshold of 12 and 15 out of 24. As these are ultimately arbitrary metrics, this system is not meant to make any definitive claims but rather to highlight areas of interest. The 208 high-risk census tracts and 1,692 census tracts provide a starting point for further analyzing housing insecurity. There are several important observations to be made from this thresholding method. First, high and medium-risk census tracts have higher African American and Hispanic/Latino populations as well as smaller white populations. There are more people with a high school diploma and fewer people below the poverty line in low-risk census tracts. High and medium-risk census tracts have similar levels of high-cost mortgage and high-cost renter households but different levels of high-cost households without a mortgage. There is slightly greater usage of public assistance and supplemental security index usage in high and medium-risk census tracts. 

A second way that this research serves as a tool for researchers is the breakdown of \hhr~association rules by state found in Appendix B. This provides a succinct overview of the presence of census tracts found to be at a high risk of housing insecurity and their relationships to different risk levels. 



\section{\textit{Spatial Aspects of Housing Insecurity}}

Global Moran's \textit{I} was calculated for each state and nationwide. Additionally, local Moran's I was used on the risk levels to determine their spatial autocorrelation at the census tract level. The first notable observation from the global Moran's \textit{I} result is the strength of the key demographic variables. African Americans, Hispanics and Latinos, White, American Indian and native Alaskan, and other all have spatial autocorrelations greater than 0.5 nationwide. This offers strong support for previous research that has identified pockets of minorities in rural areas \citep{lichter_demographic_2020}.  While the percentage of rural economies that manufacturing and agriculture, forestry, fishing, hunting, and mining make up has declined in recent years, there is a significant spatial autocorrelation to both of these variables. This is reflective of the amenities-based nature of some rural economies and further enforces the role of single industry-based economies in rural areas. Economic diversity is a good thing, so the high spatial autocorrelation of these two industries is concerning for the overall economic diversity of rural areas. Another concern is the relatively high spatial autocorrelation of households that did not move, but do not have a high school diploma. This could be reflective of areas where schools have suffered due to the declines facing rural areas \citep{blank_poverty_2005}. The low average Moran's \textit{I} for residential mobility at the national level indicates that there is not significant spatial clustering of residential mobility for high-risk households. One last observation is the fairly high spatial autocorrelation for individuals in the same house with less than a high school education (0.56). This further reinforces the potential issues facing rural areas when it comes to education.

The most significant finding of the spatial analysis is the results of local Moran's I on the sector risk level variables. Table~\ref{tab:local_moran} shows that on average, there is no local spatial autocorrelation between medium-risk level census tracts across any sector. The strongest average spatial autocorrelation is for demographics, with a value of 0.14. Low-risk levels have some level of spatial autocorrelation, the highest being for economic diversity. This statistic is higher than the low-risk spatial autocorrelation for employment diversity indicating that while the industries that typically dominate an economy are spatially autocorrelated globally, there is almost the same amount of spatial autocorrelation for census tracts with both high and low economic diversity. Another concerning observation is the relatively high local spatial autocorrelation for demographics. This follows the trend of results indicating the existence of pockets of minority populations in rural areas. 


The spatial analysis conducted in this study yielded several significant findings. Global Moran's \textit{I} calculations revealed strong spatial autocorrelations for key demographic variables nationwide, highlighting the presence of minority populations in rural areas. Despite declines in manufacturing and agriculture's share of rural economies, these industries still exhibit significant spatial autocorrelation, underscoring the persistence of single-industry-based economies in rural regions. Concerns arise from the high autocorrelation of households lacking high school diplomas, possibly indicating educational challenges in areas facing rural decline. Additionally, low residential mobility autocorrelation suggests limited spatial clustering of high-risk households' movements nationally. Local Moran's \textit{I} analysis examined the relationship between risk levels, showing negligible spatial autocorrelation among medium-risk level census tracts across sectors but significant spatial autocorrelation between high and low-risk census tracts. These findings collectively underscore the multifaceted nature of spatial patterns in rural areas, highlighting the need for targeted interventions to address economic, demographic, and educational disparities.


The spatial analysis conducted in this study has shed light on significant patterns of spatial autocorrelation at both national and local levels. The presence of strong spatial autocorrelations for key demographic variables underscores the prevalence of minority populations in rural areas, while the persistence of spatial autocorrelation in industries such as manufacturing and agriculture highlights the enduring influence of single-industry-based economies in rural regions. Concerns about educational challenges in areas facing rural decline were also raised, particularly evident in the high autocorrelation of households lacking high school diplomas. Furthermore, the limited spatial clustering of high-risk households' movements nationally suggests a nuanced understanding of rural migration patterns. Particularly noteworthy are the findings from local Moran's I analysis, which revealed significant spatial autocorrelation between high and low-risk census tracts across sectors, underscoring the complex spatial dynamics at play in rural areas. These findings emphasize the importance of targeted interventions to address economic, demographic, and educational disparities in rural communities.


\section{\textit{The Predictability of Housing Insecurity Risks}}

An important question in the study of housing insecurity is the extent to which the risk level of a census tract can be predicted. This provides insights into the variation of rural areas identified in the literature \citep{cromartie_defining_2008}. MLR regression models were used to predict the risk levels of each sector for each state and nationally. The results show very low probabilities of the model predicting the correct cluster, and very low accuracy. Most notable for the national models shown in Chapter 4's confusion matrices is their tendency to over-classify census tracts as low risk for every sector. One explanation for this is class imbalance. Class imbalance in multinomial logistic regression occurs when there are unequal proportions of observations across the different outcome categories. Figure~\ref{fig:cluster_dis} shows that there are class imbalances mostly due to high levels of lower risk levels. This could have led the models into overclassifying census tracts as low-risk when they are not. Most surprising is the lack of accuracy in the national models. Being tested using in-sample evaluation, the models should have performed notably better than the test-train split state models. The lack of accuracy of these models echoes what the literature has said: "rural" is not a blanket term but rather, it encompasses a wide-ranging and varying group of areas and people \citep{cromartie_defining_2008}. 

\section{\textit{Limitations}}

There are three significant limitations to this work. First, Due to the urban-centric lens toward housing insecurity, there is little previous research to compare to this study. \citet{gleason_using_2021} applied similar spatial techniques to census tracts in Maine and found that poverty, unemployment, and high housing costs are common in rural and urban areas of Maine. The second limitation is due to the lower rate of ACS sampling in rural areas, the accuracy of the data is limited in how well it represents the real world. While the estimates are “likely reasonable approximations of the populations they represent”, small area estimates like census tracts used here have issues with attribute uncertainty \citep{spielman_patterns_2014}. Despite this, it is currently the most detailed source of data available for rural areas. The final limitation is that as the risk-level assignment system is relative, it cannot be used to make definitive claims about the housing insecurity risk of an area.  

\endinput