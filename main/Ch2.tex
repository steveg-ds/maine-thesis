\chapter{Background}	%Chapter title

% TODO: I think i need to make custom references for some of the non-academic sources like the housing assistance council. they won't work with zotero. 

% TODO: I need to figure out which desmond_forced_2015 citations belong to the two different articles  
There are three primary areas where the extant literature must be analyzed. First,  it is important to understand the distinction between homelessness and housing insecurity. Next, it is necessary to take a multi-disciplinary look at each pillar of the 4 C's of housing insecurity. Finally is the unique challenges that face rural areas in order to better inform the theoretical framework. Together, these 3 aspects form the theoretical basis for exploring rural housing insecurity.

\section{\textit{Housing Insecurity}}

Housing insecurity is a term that stems from the shift in homelessness research from focusing on only the housed and the unhoused. %The Oxford Dictionary defines housing insecurity as "the state of not having stable or adequate living arrangements, especially due to risk of eviction or because one lives in unsafe or uncomfortable conditions" \hl{(Source)}.
 \citep{deluca_housing_2022} argue that the term housing insecurity is a more dynamic concept than the traditional housed/ unhoused binary. "Housing insecurity operates through multiple mechanisms-including material hardship, stress, environmental and infectious disease exposures, social network disruption and barriers to healthcare- to produce adverse health outcomes over the life course" \citep{leifheit_building_2022}. Housing insecurity as an area of study can be seen as a natural evolution stemming from researchers moving beyond the housed and unhoused binary. Homelessness is generally attributed to poverty and a lack of access to affordable housing \citep{noauthor_rural_2009}.  As researchers shifted away from the housed and unhoused binary, identifying characteristics that distinguish the housed from the unhoused began to identify factors that occur on the path to homelessness \hl{(Phalen and Link, 2011)}.  The identification of these factors likely contributed to the rise of housing insecurity as a term and as a field of study.  DeLuca and Rosen (2022) argue that housing insecurity may be a more useful term than the housed/unhoused binary because it acknowledges housing hardship and housing risk as a continuum that affects a wider section of the population than literal homelessness.  One issue with the study of housing insecurity is that, similar to the concept of rurality, a wide variety of definitions are used. Cox et al. (2019) analyzed 106 studies and found that current approaches to housing insecurity have three major issues: a lack of a uniform definition, it is often applied as an underdeveloped concept, and it is often measured inconsistently. An explanation for this variation is that housing insecurity operates through a variety of mechanisms \citep{leifheit_building_2022}.  Rather than giving a strict definition, it may be more beneficial to look at the dimensions of housing insecurity. \citet{cox_road_2019} identifies seven dimensions of housing insecurity: housing stability, housing affordability, housing quality, housing safety, neighborhood safety, neighborhood quality, and literal homelessness. To adequately address housing insecurity it is necessary to have a theoretical framework that can encompass these different dimensions to avoid these pitfalls. 

\section{The 4 C's of Housing Insecurity}

To understand housing insecurity in the context of the 4 C's framework, the following subsections detail each pillar of housing insecurity under the model proposed by \citep{hernandez_housing_2019}. As each pillar forms a web rather than separate pieces, there is a significant amount of overlap between pillars. The pillars meet the dimensions identified by \citet{cox_road_2019}: housing stability (consistency), housing affordability (cost), housing quality/ housing safety (conditions), and neighborhood safety/ neighborhood quality (context). 
 
\subsection{\textit{Cost}}
Housing costs are generally conceived as the amount of a households budget that goes to housing. This includes rent or mortgage payments and utilities. A cost-to-income ratio is the most common way of measuring housing affordability. It is difficult to determine one number that determines when a household is spending too much on housing. The threshold for housing affordability has varied between 25 and 50 percent with the current standard set at 30 percent \citep{kropczynski_insights_2012}.  Housing is considered affordable if the household spends less than 30 percent of its income on housing and 50 percent or more is considered a high-cost burden (\citealp{braveman_housing_2011};\citealp{swope_housing_2020};\citealp{weicher_housing_2006}). Inherent to a cost-to-income ratio is the understanding that there are other expenses necessary for survival \citep{herbert_measuring_2018}. 
%Housing costs are determined by the rate of household formation and household attrition \citep{pendall_future_2016}.  
Housing affordability affects individuals, families, and communities while access is largely determined by their demographic characteristics  (\citealp{braveman_housing_2011};\citealp{yadavalli_comprehensive_2020}. Housing affordability is directly related to residential stability and has the potential to harm both those being forced to move, the community they are leaving, and the community they are entering \citep{desmond_forced_2015}. Access to affordable housing affects the physical and material comfort of the population as well as the individual\sout{(Kumar, 2003)}. If a household cannot afford to live in their current place, they may be forced to relocate seeking more affordable housing voluntarily or through eviction and foreclosure.  If too much of a household’s money goes to housing, they may be forced to go without other necessities \citep{herbert_measuring_2018}. Those with high housing costs may also experience food insecurity as food is often considered a flexible expense while housing is a fixed expense (\citealp{fletcher_assessing_2009};\citealp{kropczynski_insights_2012}). This is only one area where low-income households may have to compromise in order to maintain their fixed housing costs. Housing is often the biggest expense for low-income families, forcing them to make trade-offs between housing and other necessities \citep{desmond_housing_2015}. The shortage of affordable housing drives lower-income families to substandard housing in worse neighborhoods \citep{braveman_housing_2011}. This creates the potential for a spiral where housing instability cannot be escaped due to the added costs of moving .\citet{kang_severe_2021} characterizes housing instability as a by-product of the affordable housing shortage wherein households can be destabilized by minor financial shocks. These factors can create a situation where housing costs lead to residential instability, which is linked to a variety of adverse conditions, especially in children and adolescents \citep{desmond_forced_2015}. Housing affordability is both influenced by and exerts influence on many other aspects of life, and its relationship to housing insecurity cannot be understated. 

\subsection{\textit{Conditions}}

Internal housing conditions have been identified as a significant factor on health (\citealp{braveman_housing_2011};\citealp{metzger_fair_2017}; \citealp{swope_housing_2020}. In one study, decent housing was found to be a more important determinant of health than education or income \citep{angel_housing_2014}. Previous environmental health research has identified five broad categories in which housing conditions contribute to adverse health effects: physical conditions, chemical conditions, biological conditions, building and equipment conditions, and social conditions \citep{jacobs_environmental_2011}. Links to an increase in disease have been tied to poverty, poor housing, and degraded environments reflecting the interconnectedness of housing insecurity issues \citep{rauh_housing_2008}.  \citet{angel_housing_2014} found that the probability of facing a chronic disease increases when housing problems accumulate and that poor housing conditions quickly degrade subjective health. These problems are amplified in the modern world where individuals spend an estimated 90 percent of their time indoors \citep{palacios_impact_2021}. The relationship between housing conditions and poverty and the range of categories that contribute to housing conditions emphasize the importance of viewing housing as a matter of health.Housing conditions also play a role in residential mobility as \citet{desmond_housing_2015} place decent housing and affordable housing as fundamentally connected and the previously mentioned rise in housing cost has not brought an increase in housing quality. The impacts of housing conditions on health means that adequate housing is a public health issue \citep{matte_housing_2000}. Despite housing conditions playing such a significant role in modern life, there is not a significant sense of communal benefit and responsibility when it comes to housing \citep{jacobs_environmental_2011}.  Without a sense of communal benefit towards housing, this leaves marginalized populations that are more likely to be exposed to harmful housing conditions without community support \citep{swope_housing_2020}. Growing a sense of communal benefit towards housing would be beneficial for all aspects of housing insecurity. 

\subsection{\textit{Consistency}} 

Residential mobility is a complicated subject because, as a broad concept, it is conceived as a good thing. That one can pack up and go somewhere with more opportunity is considered a part of the American “mystique” \citep{molloy_internal_2011}. An average of 15 percent of Americans move every year and 25 percent move over two years \citep{bachmann_ins_2014}. Classic urban economic theories hold that households make trade-offs between proximity to jobs and housing prices \citep{hu_housing_2019}. This puts low-income households at a disadvantage as their access to jobs may be lower than their wealthier counterparts. Consistency or residential stability plays an important role in the physical and social well-being of individuals, families, and communities. It has been linked to a variety of adverse conditions and affects the neighborhoods being entered and left. It has been identified as a more important predictor of community health than standard factors like poverty and racial composition (\citealp{desmond_forced_2015};\citealp{desmond_housing_2016}, \citealp{rauh_housing_2008}). An important distinction must be made between voluntary and involuntary moves \citep{siskar_who_2019}. While most moves are voluntary, millions of low-income households struggle to maintain housing stability (\citealp{phinney_exploring_2013};\citealp{kang_why_2019}). Outside of voluntary moves foreclosure, eviction, and condemnation are all drivers of forced relocation (\citealp{phinney_exploring_2013};\citealp{siskar_who_2019}). It is linked to an increase in residential instability and households forced to move often end up in places with greater disadvantage and are more likely to face additional moves \citep{desmond_forced_shell_2015}. One issue with the study of residential mobility is the limited scope of predictors that have been linked to it \citep{kang_why_2019}. One group at a particularly high risk of housing instability are those who rent their housing. Renters are particularly vulnerable to relocating to worse neighborhoods than the ones they are exiting \citep{desmond_forced_2015}. Residential instability is closely related to housing affordability, reinforcing the idea that housing insecurity is an interconnected web.

\subsection{\textit{Context}}

Context revolves around neighborhood and community characteristics including demographics, green spaces, education, and healthcare among other things. While it is impossible to capture context in its entirety, this thesis focuses on demographics, employment, housing type, and household factors as these have all been studied as matters related to housing insecurity that do not fall directly into the other pillars. The following is an interdisciplinary review of how these selected factors affect housing insecurity.  

\subsubsection{\textit{Employment}}

In the United States, the labor market is the result of cumulative individual behaviors including geographical migration and educational investments \citep{wiener_labor_2020}. The demand for labor is driven by firms, which must consider a wide variety of factors in deciding location \citep{partridge_persistent_2007}. In recent decades, the United States labor market has entered a risk regime job market where workers hold a greater share of the risk in an employment system without the perceived promise of security and stability, which has become embedded in American social and political institutions \citep{lowe_perceived_2018}. It is agreed that the Fordist regime that brought unprecedented prosperity in the early 20\textsuperscript{th} century came to an end in the 1970s \citep{stockhammer_stylized_2008}. Since this shift, the productivity of the average worker has increased by 64.6 percent while hourly pay has only increased an average of 17.3 percent between 1979 and 2021 \citep{productivity-pay-gap_2022}. Over this same period, HUD data show that the median price of a new single-family home increased from \$60,600 (\$232,091 adjusted for inflation) in Q1 of 1979 to \$369,800 in Q1 of 2021) \citep{us_census_bureau_median_1963}. These shifts in the housing market are one of the underlying factors in the rise of the affordable housing shortage. The Great Recession has had a lasting impact on the housing market within the United States. As the economic recovery did not benefit all households equally, wealth inequality has grown along both racial and ethnic lines \citep{kochhar_wealth_2014}. As wages have failed to keep up with the price of housing, the current economic system under this risk regime places those with low incomes in a precarious situation for housing affordability and residential stability. Thus, employment insecurity and income inequality are two pressing issues the United States is facing that have serious impacts on communities. “housing insecurity has risen in relative lockstep with employment insecurity” \citep[48]{desmond_housing_2016-1}. Economic conditions play a significant role in housing insecurity because adequate income, usually through employment, is critical for all aspects of housing insecurity. 


One significant cause of employment insecurity is a lack of economic diversity, generally caused by a lack of economic development. \citet{sherrieb_measuring_2010} identify three key elements connected to economic development: the level of economic resources, the level of equality in resource distribution, and the level of diversity in economic resources. Economic development alongside demographic change in rural areas has been linked to the quality and condition of local housing infrastructure \citep{barcus_heterogeneity_2011}. How policies shape economic development has a direct affect on the overall housing insecurity risk of rural communities. Amid the recent major economic shifts, globalization and shifting employment sectors play a critical role in the development path of communities which has an inherent effect on the people who live there \hl{Harrison et al., 2018}. Demonstrating the interconnectedness of communities, regional economic development in one area can encourage economic stability of its neighboring regions as well so It is important to view communities as interrelated rather than separate entities \citep{chen_economic_2018}. \citet{deller_spatial_2016} highlight the importance of economic diversity, a vital aspect of economic development, finding that more diverse economies enhance economic stability. As an insulator against economic instability, employment diversity is a key factor that policy-makers and scholars should consider as part of a holistic approach to housing insecurity. 

\subsubsection{\textit{Housing, race, and poverty}}

Housing is affected by a variety of social, political, and economic factors. “The ability of residents to access affordable housing, whether renting or buying, is in large part determined by their demographic characteristics, such as income, race, age, and educational attainment” \citep[115]{yadavalli_comprehensive_2020}. While unpredictable events may narrow the disparities, “As a rule, a household’s vulnerability to displacement should be shaped in a predictable fashion by those characteristics that define its members’ position in the [social] stratification system” \citep[5]{lee_forced_2020}. This vulnerability is driven by a combination of individual and socio-demographic factors. One major factor that has made minorities vulnerable to housing insecurity is discrimination in housing. Although the federal government took a direct interest in promoting home ownership in 1933, racial discrimination in the housing market was not outlawed until 1968 and enforcement of the law remained difficult until the Fair Housing Act of 1988 \citep{sharp_emerging_2014}. For example, the practice of redlining made it difficult for Black Americans to receive mortgages under federal aid programs and creating racial segregation that can still be seen today. At the county level, the probability of living in affordable housing decreases as the white population decreases (Brooks, 2022). In addition to racial segregation, income segregation must be considered for a holistic discussion of housing insecurity. High concentration of poverty may exacerbate housing condition issues due to a lack of revenue to maintain the necessary services at the household and local government levels. Minorities are also at a disadvantage in income segregation with poor whites being less segregated from their non-poor counterparts \citep{lichter_ruralurban_2021}. As a home is often a household's greatest source of wealth, the disadvantages minorities have in terms of housing are compounded as social and economic inequality are reproduced as these disparities continue \citep{krivo_housing_2004}.  

\subsubsection{\textit{Housing Type}} 

While owning a home is considered a part of the “American Dream,” many households rent their housing by choice or by necessity. While the many benefits of home ownership portray it as a means to a better life, renting is not inherently bad and may provide better opportunities for households that can afford it, but there are many potentially destabilizing consequences of high-cost renting \citep{drew_believing_2014}. Nationally, median rent in a poor neighborhood is \$298 compared to \$225 in a middle-class neighborhood or \$250 in an affluent neighborhood after regular expenses are deducted despite property values typically being much higher in middle-class or affluent neighborhoods \citep{desmond_poor_2019}. This creates a compounding factor for the previously mentioned disparities in home ownership. Increases in household wealth and secured debt were found to decrease the likelihood of homeowners becoming renters and vice versa \citep{anderson_effect_2021}. Money paid towards a mortgage generates long-term wealth while money paid towards rent generates wealth for the property owner. Renters with high-cost housing are unable to increase household wealth through their means of housing.  In addition to whether one rents or owns their home, the type of home can play a significant role in housing insecurity. Of particular concern is unconventional housing which includes dwellings not considered long-term habitation including RVs/ campers, vans, and boats. These unconventional forms of housing may keep people off the streets, but they are not always a stable mode of housing. For RV and camper living, people who are undocumented or are unable to keep up with legal or maintenance costs for vehicles end up losing their housing \citep{wakin_not_2005}. Those who rent with a high housing cost and those who live in unconventional housing should be considered to have a high-risk of housing insecurity. In rural areas, mobile homes are often seen as an affordable option but they come with certain risks not as common in traditional housing. Rural areas face unique housing issues because one of the most common housing solutions is mobile homes. Structural problems like poor construction and risks of air pollution and fire create a unique problem \citep{mactavish_policy_2006}. Mobile homes also carry a unique set of circumstances that may put households at a greater risk of housing insecurity and are found frequently in rural areas. Mobile homes and the land they are situated on can be either owned or rented. It is common in mobile home parks for households to own their home but not the land it is on. Key issues with mobile homes include their financing: typically done through more expensive but easier obtained means than a mortgage such as personal property or chattel loans; mobile homes do not build wealth in the same way as they typically depreciate rather than appreciate; households on rented land have little control over their length of stay; they also tend to have worse construction and higher risks of air pollution and fire than traditional homes \citep{mactavish_wrong_2007}. 


\subsubsection{\textit{Household income, aid, and Transportation}}
In his first State of the Union address, President Lyndon B. Johnson asked Congress to declare an “unconditional war on poverty… not only to relieve the symptom of poverty but to cure it and, above all, to prevent it.” % source 
Since then, a patchwork of programs regulated at the federal, state, and local levels has arisen. A large part of the federal government's growth in the late 20\textsuperscript{th} century is from the expansion of social welfare spending \citep{fishback_social_2020}. Today, The primary mechanism of income distribution is what \citet{berkowitz_gaps_2023} refer to as the “factor payment system” in which those who work and those who own the means of production and one’s relation to this system and the labor market are closely related to one’s poverty risk. To alleviate this poverty risk, social programs that utilize different mechanisms are available to those who qualify. These mechanisms can be divided into categorical and income-targeted policy designs, alongside decentralization, where some receive benefits based on “demographically defined, categorical eligibility structures” and others enjoy standardized federal assistance through social insurance with some qualifying for income-based or “means-tested” programs \citep{bruch_poverty_2023}. Households must fall below certain income and asset thresholds to qualify for means-tested programs \citep{rank_welfare_2002}. For housing, there is a wide variety of housing policies and programs aimed at low-income individuals. These take the shape of voucher programs by subsidizing privately held property although some recipients live in public housing \citep{kim_housing_2017}. For rural areas, the U.S. Department of Agriculture (USDA) has a variety of programs aimed at improving living conditions in rural areas including direct or guaranteed loans for single or multi-family housing, and infrastructure programs for water, electricity, and telecommunications \citep{noauthor_usda_2023}. Transportation plays a large role in social and economic life. Access to everything from education to healthcare depends on the infrastructure and the ability to use available means of transportation. Rural areas often do not have public transportation, leading residents to depend more on automobiles. An analysis of 2009 National Household Travel Survey data found that 72 percent of households with a yearly income of \$20,000 have access to a household vehicle compared to over 97 percent of households making \$50,000 \citep{blumenberg_automobile_2012}. Automobile ownership can be a crucial factor in avoiding residential instability \citep{kang_why_2019}. Households are twice as likely to be auto-deficient (less than 1 car per driver) than zero-vehicle households where a vehicle is not needed \citep{blumenberg_car-deficit_2020}. This is a concerning for rural areas without public transportation where distances may be too far or too dangerous for alternate means of transportation due to a lack of proper road infrastructure like bicycle lanes and sidewalks.
 
\section{Challenges for Rural Areas}

 Rurality is often defined simply as not being urban \citep{robertson_rural_2007}. Defining rural areas in contrast to urban areas largely excludes the variation between rural areas. The Census Bureau defines metro areas as urban areas of 50,000 people or more, and urban clusters of 2,500 to 49,999 people with all other areas classified as rural; the Office of Management and Budget defines metro areas as urban cores with populations of 50,000 or more people, micro areas as urban cores of 10,000 to 49,999 people where micro areas and counties outside of metro and micro areas are considered rural \citep{health_resources__services_administration_defining_2022}. % probably need more definitions to prove my point
 The lack of universally accepted definitions of rurality reduces the amount of time and resources that can be dedicated to struggling communities \citep{yousey_defining_2018}. 
 

 Part of the blanket construct of rural areas is that they are cheaper to live in. However, \citep{kurre_is_2003} note that there is relatively little systematic data that supports this presumption. Rural areas face the same low per capita income and poverty problems faced by urban areas \citep{castle_place_2011}. \citep{zimmerman_does_2008} found no consistent pattern of lower prices across all of the rural counties in Pennsylvania. While the dollar amount paid for housing may be lower, given the different socio-economic circumstances of rural areas, housing costs alone may not fully encapsulate the situation \citep{kropczynski_insights_2012}.  While there is limited research on homelessness in rural areas, previous research has documented the unique struggles of rural areas that should be addressed in a discussion on rural housing insecurity. First, previous research has identified both pockets of prosperity and pockets of deep poverty in rural areas. Concentrated poverty is "often the manifestation of an interactive and inter-generational dynamic between structural changes that restrict economic opportunities and the emergence of populations with characteristics that put members at a relatively high risk of poverty" \citep[?]{thiede_spatial_2018}. Poverty is acknowledged more in urban areas, but poverty rates are highest in both remote rural counties and in cities(\sout{(Miller and Weber, 2007)};\citealp{crandall_local_2004}). Persistent poverty, typically defined as poverty levels above 20 percent, is geographically concentrated in rural regions \citep{crandall_local_2004}. In 2010, the poverty rate among the rural population was higher than that of the nation overall \citep{burton_inequality_2013}. \citet{lichter_rural_2011} found that 40.5 percent of high-poverty places are in high-poverty counties for non-metro areas and the poor and non-poor are becoming increasingly segregated, with higher concentrated poverty among minorities. A cluster analysis found that of 3,017 places which is about 5 percent of the nation's population experience persistent poverty and 84 percent of this population lives in rural rather than urban areas \sout{(Peters 2009)}. \citet{lichter_changing_2007} found that 85 percent of the nearly 500 counties with poverty rates over 20 percent and the 12 counties with poverty rates over 40 percent are in non-metro areas. The areas with persistent poverty have some similar characteristics: they have primarily agricultural or resource-based economies, reduced employment opportunities due to economic changes, or gentrification is making living costs unaffordable for many people \citep{robertson_rural_2007}. One potential explanation for the persistent effects of poverty in rural areas is the isolation from schools, services, social interactions, and labor market resources \citep{canto_rural_2014}. One explanation for the high rates of poverty in rural areas is their isolation. Isolation stems from limited ease of travel or access to nearby markets and population centers which can hinder economic development, meaning that greater geographic isolation is associated with both lower income and greater poverty rates \citep{blank_poverty_2005}. 
 
 Looking only at poverty does not tell the full story of rural areas. There are more than 300 rural counties spread across the nation that are more "prosperous" than the rest of the nation based on measures spanning education, housing, poverty, and unemployment \citep{isserman_why_2009}.  This highlights the need for an approach to rural areas that is relative rather than absolute. \citep{metzger_fair_2017} highlight the tendency for Americans to segregate themselves not only based on race but on class too. A tendency for the rich and the poor to cluster around themselves could explain these findings in rural areas. This spatial inequality is critical to understanding rural poverty \citep{thiede_spatial_2018}.  Spatial inequality expands concerns with stratification into the realm of geographic space \citep{lobao_spatial_2002}. In rural areas where location determines many aspects of the community constructed on top of it, researchers cannot ignore the implications of spatial inequality. That there are both highly prosperous and high poverty rural areas indicates a need for a better understanding of the role of spatial inequality. 

 Another problem that rural areas are facing is a growing economic divide between urban and rural areas \citep{bjerke_mover_2019}.  Rural communities have been hit hard by economic changes in recent decades, driven by the transition from a production to a consumption-based economy \citep{pendall_future_2016}. During this shift, employment became increasingly scarce for agricultural workers \citep{kropczynski_insights_2012}. Today, manufacturing is responsible for 21 percent of rural non-agricultural earnings \citep{low_rural_2017}. Economic development is therefore a fundamental issue to rural areas. While manufacturing has grown, the majority of counties that experienced manufacturing employment growth between 2001 and 2015 had low levels of growth in terms of total employment \citep{low_rural_2017}. \citep{blank_poverty_2005} note that rural areas often have more limited job opportunities and are more likely to rely on one industry rather than having a diversified economy. Preventing the amelioration of problems facing rural areas is the relatively uncoordinated approach to rural development that has occurred despite the active role the federal government has played in it \citep{wilson_rural_2016}. As a result some rural regions have experienced periods of sustained growth while others have faced the previously mentioned issues \sout{(Johnson, 2012)}. One aspect of this is the friction that is created when rural households are too distant from adequate labor markets that enable them to support their families \citep{sparks_poverty_2013}. This has created a common migration pattern where many people move to urban areas for greater economic opportunities leaving rural towns with a smaller, older population and a less skilled labor force \citep{bjerke_mover_2019}. The effects of these population decreases span across socioeconomic factors. School consolidations, reductions in local services, closed businesses, increased infrastructure costs, poorer schools, poorer healthcare, and limited public services have all been tied to population shrinks and communities have little ability to control these processes that limit economic mobility and can perpetuate poverty \citep{martinez_rural_2021}. There is a cyclical nature to the problems facing rural areas. For the areas affected by poverty, it becomes difficult for systemic improvements because the economic decline inherently reduces the resources available in the community for addressing the issues at hand.

 Rural areas face significant consequences for the historical forces that shape housing today. When discussing rural poverty it must be noted that there is an underlying assumption that the dynamics of poverty are fundamentally different from urban areas \citep{thiede_spatial_2018}. Persistent problems faced by the rural poor include "physical isolation and poor public transportation, inadequate schools, and limited access to medical care and other basic public services while institutional support services are frequently limited or simply unavailable" \citep[?]{lichter_changing_2007}. Part of this is driven by the outflow from rural areas to urban areas. Rural areas have seen a population reduction, reducing the capabilities of public services to accommodate those in need \citep{bjerke_mover_2019}. As mentioned earlier, there are a variety of reasons why households move. \citet{thiede_spatial_2018} found that from 2000 to 2012, increases in poverty were larger in rural counties than urban counties with the highest increases in exposure and the rural black population was by far the most disadvantaged over this period. Rural areas are not as diverse as the United States overall, and many rural minorities are geographically central in regions tied to historical and economic dynamics \hl{(Housing Assistance Council, 2012)}. Another demographic group that is significant to rural areas is Hispanics and Latinos, despite the widespread population decline of rural areas \citep{lichter_demographic_2020}. African Americans and Hispanics and Latinos face similar discrimination in the housing market with the benefits of housing are dramatically smaller for these demographics \citep{krivo_housing_2004}. Thus, the pockets of these groups in rural areas should be considered to be at a higher risk of housing insecurity due to the effects of these historical forces. 

 
 
\section{\textit{Summary}}

Throughout this chapter, the 4 C's of housing insecurity have been covered. It is important to highlight the interconnected nature of the 4 C's. There is a significant overlap between each pillar of housing insecurity. Housing costs, housing type, and housing conditions are necessarily linked to the economic conditions of a household. These economic conditions are linked to the household wage/ aid factors that encapsulate their economic status. One's relation to the poverty level and education plays a significant role in housing accessibility and these factors are intrinsically linked to the context that they live in. Rural areas face numerous issues, some that align with problems in urban areas and some that do not such as the presence of mobile homes and economies built around single amenities. Pockets of persistent poverty and prosperity. Any discussion on housing insecurity must consider the historical forces affecting modern-day race and poverty, and these forces relate to all aspects of life. When taken as a web, this model encompasses the wide-ranging socio-economic factors that surround housing insecurity. 

\endinput