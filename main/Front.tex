\copyrightpage[Steve Garcia]{2024}		% Optional, comment out or delete if undesired

\begin{abstract}
This paper explores rural housing insecurity through Swope and Hernandez’s (2019) 4 C's of housing insecurity in rural areas. Rural census tracts are defined by the United States Department of Agriculture (USDA) Rural Urban Code Continuum (RUCA) codes seven through ten to avoid an overly restrictive definition.  Little attention has been paid to rural areas in the conversation on housing, to facilitate further discussion this exploratory study uses unsupervised machine learning to group census tracts into risk levels across 7 sectors of ACS data based on housing insecurity factors found in the literature. Multinomial logistic regression is used to determine variation between states based on how well state risk levels can be predicted with the national dataset. Additionally, spatial autocorrelation is used to analyze how spatially clustered the risk levels and housing insecurity risk variables.  The results indicate that many rural census tracts have a medium risk of housing insecurity, and the risk levels are hard to predict. The spatial autocorrelation results show that the housing insecurity variables are not as highly spatially clustered as expected.  
\end{abstract}

\begin{layabstract}{Housing Insecurity, homelessness, data mining}	% Replace the ... with the list of keywords
This paper explores rural housing insecurity through Swope and Hernandez’s (2019) 4 C's of housing insecurity in rural areas. Rural census tracts are defined by the United States Department of Agriculture (USDA) Rural Urban Code Continuum (RUCA) codes seven through ten to avoid an overly restrictive definition.  Little attention has been paid to rural areas in the conversation on housing, to facilitate further discussion this exploratory study uses unsupervised machine learning to group census tracts into risk levels across 7 sectors of ACS data based on housing insecurity factors found in the literature. Multinomial logistic regression is used to determine variation between states based on how well state risk levels can be predicted with the national dataset. Additionally, spatial autocorrelation is used to analyze how spatially clustered the risk levels and housing insecurity risk variables.  The results indicate that many rural census tracts have a medium risk of housing insecurity, and the risk levels are hard to predict. The spatial autocorrelation results show that the housing insecurity variables are not as highly spatially clustered as expected.  
\end{layabstract}

\begin{dedication}
Dedicated to Steve Mauldin, who taught me to find love in the emptiness of the absurd 
\end{dedication}

\begin{acknowledgements}
I would like to thank the following people for helping with this research project:
Thank you to my advisor, Dr. Matt Dube for providing guidance and feedback throughout this project, To my wife Katie for encouraging to pursue my dreams, to Dr. Jeremy Mhire for teaching me how to think, and Dr. Dolliann Hurtig for teaching me how to live. 
\end{acknowledgements}

\pagebreak

% Commands for the required lists
\tableofcontents

\pagebreak

\listoftables				% Include only if there are tables in the thesis
\pagebreak
\listoffigures		
\pagebreak		% Include only if there are figures in the thesis


%\listof{equations}{List of Equations}

% Sets the document spacing and pagestyle.
\mainmatter

\endinput
