\chapter{Addressing Rural Housing Insecurity}	%Chapter title

\section{\textit{Defining Rurality}}
Rather than strictly defining rurality, this thesis uses the United States Department of Agriculture (USDA) Rural-Urban Continuum spectrum. The following codes are used to encapsulate rurality:

\begin{table}[htbp]
    \centering
    \caption{RUCA Codes}
    \label{tab:town_description}
    \begin{tabular}{|c|p{10cm}|}
        \hline
        \textbf{Number} & \textbf{Description} \\
        \hline
        7 & Small town core: primary flow within an Urban Cluster of 2,500 to 9,999 (small UC) \\
        \hline
        8 & Small town high commuting: primary flow 30 percent or more to a small UC \\
        \hline
        9 & Small town low commuting: primary flow 10 percent to 30 percent to a small UC \\
        \hline
        10 & Rural areas: primary flow to a tract outside an urban area or urban cluster \\
        \hline
    \end{tabular}
\end{table}
\hl{cite ruca codes}

The range of RUCA codes described in Table 1.1 was chosen to be inclusive rather than exclusive, including small towns with various levels of commuting to urban clusters and areas classified as rural. We include small towns because they often serve as hubs for rural areas, serving an important role in rural areas and \hl{source} has identified a significant amount of rural people that live on the edge of urban places, like small towns. Spatial autocorrelation is used to determine how often similar rates of each variable occurred across each rural census tract in each state. Finally, multinomial logistic regression is used to determine how well the risk levels of a census tract can be predicted based on the nationwide dataset.  All analysis was conducted in the R statistical language.  

\section{\textit{Applying the 4 C's}}
Applying the four C's of housing insecurity necessitates a mix of quantitative and qualitative analysis. To use the model to classify areas into risk levels, it is necessary to define thresholds for each pillar based on the literature review. For housing costs, an area is at a higher risk of housing insecurity as the number of households spending more than 30 percent of their income on housing increases. A study of urban housing insecurity would place a higher emphasis on renters with high housing costs, but the extent renting versus owning affects housing insecurity in rural areas is unknown. Housing Conditions are difficult to encapsulate because they encompass a broad range of factors. An additional challenge is a lack of rural-specific housing conditions data. This thesis measures housing conditions by the lack of complete plumbing and kitchen facilities, with the assumption that if these are missing, there are likely other factors the household is struggling with as well. The risk of housing insecurity in an area therefore increases as the number of occupied and unoccupied housing lacking these fundamental needs increases. Priority is given to occupied housing as there will always be some amount of housing not fit for habitation. Consistency, or residential mobility, is difficult to encapsulate because many households move for reasons unrelated to housing insecurity. To focus on the subset of households that are at a high risk of becoming housing insecure, the scope of residential mobility is limited to those who have moved in the past year without a college degree or are below or just above the poverty line. These groups are more likely to move to more precarious situations than those making moves for economic and social reasons unrelated to housing insecurity. Context is the most difficult pillar of the four C's to capture because it encapsulates many individual, social, and political factors. Six different sets of factors are used to capture context. Due to the influence of social, political, and historical processes, demographic diversity is used to capture the effect that race has on housing insecurity risk. The previously mentioned measures of residential stability also contribute to the context of an area, encapsulating education and poverty. The type of housing individuals in an area live in is a significant factor of context because mobile homes, while being seen as a means of affordable housing, can signify a risk of housing insecurity when taken in tandem with other factors. The final measure in context is household factors. This range of household factors is designed to encapsulate different individual, social, and economic factors that contribute to housing insecurity 



\section{\textit{Data}}
Eight different sectors of 2019 ACS 5-year variables are used to capture the 4 Cs of housing insecurity using indicators of housing insecurity identified in the literature. These sectors are housing cost, housing quality, housing type, economic diversity, education mobility, poverty mobility, and household worker/aid. For demographic variables we use seven variables including an “other” variable to account for race/ ethnicity and the number of people over or under 18 by gender. The economic diversity data is the number of people employed across 13 distinct categories. It was necessary to create three compound variables: high-cost with a mortgage, high cost without a mortgage, and high-cost rent to use the standard affordability measure of 30 percent. There are four variables for housing conditions which include houses with an incomplete or insufficient kitchen or plumbing for occupied and unoccupied housing units. Two sets of variables account for residential mobility: education and poverty that include those who did not move, those who moved in and out of county and state.  Due to the housing affordability and income inequality crises, those below the poverty level and those at 125 percent of the poverty level as high risk for housing instability are included in residential mobility: poverty. For education those with and without a high school diploma are included as those without a college degree may face higher barriers to well-paying and stable employment. Wage/aid data include households without income, households that receive public assistance, households that receive supplemental security income, households with investment income, households with other income, households with 3+ workers and the household Gini index. For housing type, renters and owners of mobile homes, single family homes, small and large multi-family homes, and renters and owners of unconventional housing are included.  

\section{\textit{Data processing}}
In order to ensure the integrity of the data, census tracts that lacked specific sector-related information were excluded from the analysis. These omitted tracts were assigned a risk level of zero, a measure adopted to preserve the largest possible number of census tracts for subsequent analyses. To mitigate potential biases stemming from differences in population sizes and geographic areas, a standardized approach was employed across each sector. This involved scaling all dataset components to a common base unit. Demographic and economic diversity metrics were adjusted proportionally to the population size. Meanwhile, data pertaining to household expenses and types were scaled based on the counts of homeowners and renters. The household dataset underwent normalization corresponding to the total number of households, whereas housing condition indicators were adjusted relative to the total count of occupied and unoccupied housing units. It's essential to note that all numerical values within the dataset have been uniformly represented as percentages, except for the household Gini Index, which retains its original values.

\section{\textit{Methods}}
Supervised and unsupervised machine learning algorithms are used alongside globl and local Moran's I spatial autocorrelation, the Queen Contiguity spatial relationship algorithm to form and analyze the housing insecurity risk assignment system, and multinomil logistic regression to examine the predictive abilities of the risk assignment system.

\subsection{\textit{Neighbors Algorithm}}

Communities often share dependencies across state lines, making it unjust to disregard neighboring communities in a state-based housing insecurity analysis. To address this, the analysis encompasses census tracts within 15 miles of a state's outermost tract. Any census tract sharing a boundary with a tract within this range is considered, ensuring a more inclusive evaluation of rural housing insecurity. This process is repeated for each state in the continental United States.

The formula for queen contiguity neighbors is shown in Equation \ref{eq:queen_neighbors}.

\begin{equation}
    \label{eq:queen_neighbors}
    \begin{aligned}
        \text{Top-left:} & \quad (x-1, y-1) \\
        \text{Top:} & \quad (x, y-1) \\
        \text{Top-right:} & \quad (x+1, y-1) \\
        \text{Left:} & \quad (x-1, y) \\
        \text{Right:} & \quad (x+1, y) \\
        \text{Bottom-left:} & \quad (x-1, y+1) \\
        \text{Bottom:} & \quad (x, y+1) \\
        \text{Bottom-right:} & \quad (x+1, y+1)
    \end{aligned}
\end{equation}

\subsection{\textit{K-Medoids Clustering}}
K-medoids clustering is a partitioning technique aimed at dividing a dataset into \(K\) distinct and non-overlapping clusters. Unlike K-means clustering, which utilizes centroids as cluster representatives, K-medoids employs actual data points within the dataset as cluster representatives. The key advantage of K-medoids lies in its robustness to outliers and noise due to its use of real data points. The objective of K-medoids clustering is to minimize the sum of dissimilarities within clusters. Each state, including neighboring census tracts, is clustered individually. The cluster medians are analyzed to determine which clusters have a high, medium, or low risk of housing insecurity based on the factors previously identified in the literature review. To bring areas of concern, census tracts are labeled as high-risk if the sum of their risk levels is 12 and medium-risk if the sum of their risk levels is 15. Out of a total of 24, this approach highlights the areas that show the most vulnerability across sectors.The formula for K-medoids clustering is shown in Equation \ref{eq:k-medoids}.

\begin{equation}\label{eq:k-medoids}
    \begin{aligned}
        \underset{S}{\text{minimize}} \quad & \sum_{i=1}^{K} \sum_{x \in C_i} d(x, m_i) \\
        \text{where:} \\
        S & : \text{The set of clusters.} \\
        K & : \text{The number of clusters.} \\
        i & : \text{Index representing each cluster (\(1 \leq i \leq K\)).} \\
        C_i & : \text{The \(i\)-th cluster containing data points.} \\
        x & : \text{A data point within a specific cluster (\(x \in C_i\)).} \\
        m_i & : \text{The medoid (representative) of the \(i\)-th cluster.} \\
        d(x, m_i) & : \text{The dissimilarity (distance) between data point \(x\) and medoid \(m_i\).}
    \end{aligned}
\end{equation}

\subsection{\textit{Association Rules Learning}}

Association Rules learning is a data mining technique used to uncover interesting relationships between variables in large datasets. It aims to discover patterns in the form of rules indicating the co-occurrence or association between items within transactions or events.

Association rule learning involves two main metrics:

\textbf{Support (s)}: Measures the frequency or occurrence of an itemset in the dataset.
\[
\text{Support}(A \rightarrow B) = \frac{\text{Transactions containing both A and B}}{\text{Total transactions}}
\]

\textbf{Confidence (c)}: Measures the conditional probability that an item B appears in a transaction given that item A is present.
\[
\text{Confidence}(A \rightarrow B) = \frac{\text{Support}(A \cup B)}{\text{Support}(A)}
\]

Here, association rules are used to analyze the common occurrences between sector risk levels. Of primary interest are unexpected relationships where a high-risk level is associated with a low-risk level and vice versa. 


\subsection{\textit{Moran's I}}
The Global Moran's I is a statistical measure used in spatial analysis to detect spatial clustering or dispersion of similar values within a dataset. It quantifies the degree of spatial autocorrelation by assessing whether neighboring locations exhibit similar or dissimilar attribute values. Specifically, Moran's I considers both the values of the locations and the spatial relationship between them, providing a single coefficient that ranges from -1 to 1, with 0 indicating spatial randomness. This measure helps identify patterns in spatial data, highlighting if similar values tend to be close to each other or dispersed across the study area. The Moran's I values for each variable is calculated for each state and nationally in order to analyze how the housing insecurity factors cluster in space. The formula for Global Moran's I is shown in Equation \ref{eq:moran_i}.:

\begin{equation}\label{eq:moran_i}
    I = \frac{N}{W} \frac{\sum_{i=1}^{N} \sum_{j=1}^{N} w_{ij} (x_i - \bar{x})(x_j - \bar{x})}{\sum_{i=1}^{N} (x_i - \bar{x})^2}
\end{equation}

Where:

\begin{align*}
I & : \text{Moran's I statistic, representing the degree of spatial autocorrelation.} \\
N & : \text{Total number of spatial units (e.g., census tracts, regions).} \\
W & : \text{Total spatial weight in the dataset.} \\
w_{ij} & : \text{Spatial weight between spatial units \(i\) and \(j\).} \\
x_i & : \text{Value of the variable of interest in spatial unit \(i\).} \\
\bar{x} & : \text{Mean value of the variable of interest across all spatial units.}
\end{align*}

To measure how housing insecurity risk levels cluster in space, local Moran's I is also used to indicate the spatial relationship of housing insecurity risk levels. The formula for local Moran's I is shown in Equation \ref{eq:local_moran_i}. Local Moran's I does not follow the same -1 to 1 structure of global Moran's I, but it retains the structure that positive values indicate stronger spatial autocorrelations and negative values indicate stronger spatial randomness. 

\begin{equation}\label{eq:local_moran_i}
    I = \frac{N}{W} \frac{\sum_{i=1}^{N} \sum_{j=1}^{N} w_{ij} (x_i - \bar{x})(x_j - \bar{x})}{\sum_{i=1}^{N} (x_i - \bar{x})^2}
\end{equation}

Where:

\begin{align*}
I & : \text{Moran's I statistic, representing the degree of spatial autocorrelation.} \\
N & : \text{Total number of spatial units (e.g., census tracts, regions).} \\
W & : \text{Total spatial weight in the dataset.} \\
w_{ij} & : \text{Spatial weight between spatial units \(i\) and \(j\).} \\
x_i & : \text{Value of the variable of interest in spatial unit \(i\).} \\
\bar{x} & : \text{Mean value of the variable of interest across all spatial units.}
\end{align*}


\subsection{\textit{Multinomial Logistic Regression}}

After the clustering is performed and the clusters are analyzed, each sector is assigned a new variable containing the risk levels for each census tract. Cross split validation is used wherein for each state, a new model is trained on all states except the target state. The probability that each census tract is its actual classification is preserved for the analysis. Additionally, to better understand how the \hs factors contribute to the risk levels, for each sector a national model is trained on the entire dataset so that the model can be analyzed and prediction power can be measured under the best-case scenario. The formula for multinomial logistic regression is shown in Equation \ref{eq:multinom_regression}.

\begin{equation}\label{eq:multinom_regression}
    \log\left(\frac{{P(Y = k \mid X)}}{{P(Y = K \mid X)}}\right) = \beta_{0k} + \beta_1 X_1 + \beta_2 X_2 + \dots + \beta_p X_p
\end{equation}

Where:
\begin{align*}
    & \log: \text{Natural logarithm function} \\
    & P(Y = k \mid X): \text{Probability of the outcome being in category \(k\) given predictor variables \(X\)} \\
    & P(Y = K \mid X): \text{Probability of the outcome being in the reference category \(K\) given \(X\)} \\
    & \beta_{0k}: \text{Intercept for category \(k\)} \\
    & \beta_1, \beta_2, \dots, \beta_p: \text{Coefficients corresponding to predictor variables \(X_1, X_2, \dots, X_p\)} \\
    & X_1, X_2, \dots, X_p: \text{Predictor variables} \\
    & k: \text{Specific category being predicted} \\
    & K: \text{Reference category}
\end{align*}

\endinput