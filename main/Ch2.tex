\chapter{Factors of housing insecurity}	%Chapter title
\section{\textit{Overview}}
A house is far more than four walls, a roof, and some doors, The characteristics and location of a house can make a significant impact on one’s life. In the United States, housing is often a family’s greatest expenditure, greatest source of wealth, and a place of safety and gathering \citep{braveman_housing_2011}. The federal government has long acknowledged this through legislation like the Housing Act of 1949, and social programs and development goals developed by the U.S. Department of Housing and Urban Development. Housing is often seen as one of the most fundamental determinants of health, and a lack of adequate housing can produce adverse health outcomes and acts as a foundation for “social, psychological, and cultural well-being” (\citealp[p.17]{dalessandro_housing_2020}; \citealp{leifheit_building_2022}). When it comes to addressing literal homelessness, public health experts differentiate between preventative services and reactive or emergency services (O’Regan et al., 2021). Preventive services prevent households from becoming homeless, while reactive or emergency services step in after a household becomes homeless. As homelessness is often seen as an urban problem, most intervention occurs in urban areas \citep{gleason_using_2021}. With little infrastructure for reactive services in rural areas, the 4 C’s approach to housing insecurity can highlight areas of critical concern for devoting resources to reactive services and identify areas where preventative services can improve or expand. The 4 C’s housing as an interconnected web of factors that impact health and encapsulates “this unequal distribution of housing disparities along other axes of inequality, and the historical forces shaping unequal housing opportunities” \citep[1]{hernandez_housing_2019}.  

\subsection{\textit{Cost}}
Housing affordability affects individuals, families, and communities and (Braveman et al. 2011). Housing affordability is directly related to residential stability and has the potential to harm both those being forced to move, the community they are leaving, and the community they are entering (Desmond et al., 2015). If a household cannot afford to live in their current place, they may be forced to relocate seeking more affordable housing or through eviction and foreclosure. Housing is considered affordable if the household spends less than 30\% of its income on housing and 50\% or more is considered a high-cost burden (\citealp{braveman_housing_2011};\citealp{swope_housing_2020};\citealp{weicher_housing_2006}).  If too much of a household’s money goes to housing, they may be forced to go without other necessities \citep{herbert_measuring_2018}. \citet{fletcher_assessing_2009}  note that a body of research has found that those with high housing costs may also experience food insecurity. The shortage of affordable housing drives lower-income families to substandard housing in worse neighborhoods \citep{braveman_housing_2011}. \citet{kang_severe_2021} characterizes housing instability as a by-product of the affordable housing shortage wherein households can be destabilized by minor financial shocks. These  factors can create a situation where housing costs lead to residential instability, which is linked to a variety of adverse conditions, especially in children and adolescents \citep{desmond_forced_2015}.  

\subsection{\textit{Conditions}}

Many scholars have identified internal housing conditions as a significant factor on health (\citealp{braveman_housing_2011};\citealp{metzger_fair_2017}; \citealp{swope_housing_2020}. Previous environmental health research identified five broad categories in which housing conditions contribute to adverse health effects: \textit{physical conditions}, \textit{chemical conditions}, \textit{biological conditions}, \textit{building and equipment conditions}, and \textit{social conditions} \citep{jacobs_environmental_2011}. Links to an increase of disease have been tied to poverty, poor housing, and degraded environments reflecting the interconnectedness of housing insecurity issues \citep{rauh_housing_2008}. Rural areas face unique housing issues because one of the most common housing solutions is mobile homes. Structural problems like poor construction and risks of air pollution and fire create a unique problem \citep{mactavish_policy_2006}. An area of particular concern are marginalized populations who are more likely to be exposed to harmful housing conditions \citep{swope_housing_2020}. Housing conditions also play a role in residential mobility as \citet{desmond_housing_2015} place decent housing and affordable housing as fundamentally connected and the previously mentioned rise in housing cost has not brought an increase in housing quality.  

\subsection{\textit{Consistency}} 

Mobility is conceived as a good thing, that one can pack up and go somewhere with more opportunity is considered a part of the American “mystique” \citep{molloy_internal_2011}. An average of 15\% of Americans move every year and 25\% move over the course of two years \citep{bachmann_ins_2014}. Classic urban economic theories explain that households make trade-offs between proximity to jobs and housing prices \citep{hu_housing_2019}. This puts low-income households at a particular disadvantage as their access to jobs may be lower than their wealthier counterparts. Consistency or residential stability plays an important role in the physical and social well-being of individuals, families, and communities. Residential instability has been linked to a variety of adverse conditions and affects the neighborhoods being entered and left (\citealp{desmond_forced_2015};\citealp{desmond_housing_2016}). People move for a variety of reasons, but an important distinction must be made between voluntary and involuntary moves \citep{siskar_who_2019}. Of particular concern is forced relocation, foreclosure, eviction, and condemnation are all drivers of forced relocation (\citealp{phinney_exploring_2013};\citealp{siskar_who_2019}). Forced relocation is linked to an increase in residential instability and households forced to move often end up in places with greater disadvantage and are more likely to face additional moves \citep{desmond_forced_2015}. 

\subsection{\textit{Context}}

Context revolves around neighborhood and community characteristics including demographics, green spaces, education, and healthcare among other things. It is impossible to measure context entirely, demographics, economic diversity, and household wage/ aid factors are used as these have all been studied as matters related to housing insecurity that do not fall directly into the other 4 C’s. The following sub-sections provide an interdisciplinary review of how these factors affect housing insecurity.  

\subsubsection{\textit{Employment}}
Employment insecurity and income inequality are two pressing issues the United States is facing that have serious impacts on communities. First, “housing insecurity has risen in relative lockstep with employment insecurity” \citep[48]{desmond_housing_2016-1}. In recent decades, the United States labor market has entered a risk regime job market where workers hold a greater share of the risk in an employment system without the perceived promise of security and stability, which has become embedded in American social and political institutions (Lowe, 2018). It is agreed that the Fordist regime that brought unprecedented prosperity in the early 20th century came to an end in the 1970s \hl{(Stockhammer 2008)}. Since this shift, productivity of the average worker has increased 64.6\% while hourly pay has only increased an average of 17.3\% between 1979 and 2021 \citep{noauthor_productivity-pay_2022}. Over this same period, U.S. Housing and Urban Development data shows that the median price of a new single-family home increased from \$60,600 (\$232,091 adjusted for inflation) in Q1 of 1979 to \$369,800 in Q1 of 2021) \citep{us_census_bureau_median_1963}. As wages have failed to keep up with the price of housing, the current economic system under this risk regime places those with low incomes in a precarious situation for housing affordability and residential stability.  

\subsubsection{\textit{Housing, race, and poverty}}
Housing is affected by a variety of social, political, and economic factors. “The ability of residents to access affordable housing, whether renting or buying, is in large part determined by their demographic characteristics, such as income, race, age, and educational attainment” \citep[115]{yadavalli_comprehensive_2023}. While unpredictable events may narrow the disparities, “As a rule, a household’s vulnerability to displacement should be shaped in predictable fashion by those characteristics that define its members’ position in the [social] stratification system” \citep[5]{lee_forced_2020}. Although the federal government took a direct interest in promoting home ownership in 1933, racial discrimination in the housing market was not outlawed until 1968 but enforcement of the law remained difficult until the Fair Housing Act of 1988 (Sharp \& Hall, 2014). For example, the practice of redlining made it difficult for Black Americans to receive mortgages under federal aid programs and creating racial segregation that can still be seen today. At the county level, the probability of living in affordable housing decreases as the white population decreases (Brooks, 2022). In addition to racial segregation, income segregation must be considered for a holistic discussion of housing insecurity. \citet{lichter_rural_2011} found that 40.5\% of high-poverty places are in high-poverty counties for non-metro areas and the poor and non-poor are becoming increasingly segregated, with higher concentrated poverty among minorities. High concentration of poverty may exacerbate housing condition issues due to a lack of revenue to maintain the necessary services at the household and local government levels. Minorities are also at a disadvantage in income segregation with poor whites being less segregated from their non-poor counterparts \citep{lichter_ruralurban_2021}. As a home is often a household's greatest source of wealth, the disadvantages minorities have in terms of housing are compounded as social and economic inequality are reproduced as these disparities continue \citep{krivo_housing_2004}.  

While owning a home is considered a part of the “American Dream,” many households rent their housing by choice or by necessity. Renting is not inherently bad and may provide better opportunities for households that can afford it, but there are many potentially destabilizing consequences of high-cost renting. Nationally, median rent in a poor neighborhood is \$298 compared to \$225 in a middle-class neighborhood or \$250 in an affluent neighborhood after regular expenses are deducted despite property values typically being much higher in middle-class or affluent neighborhoods \citep{desmond_poor_2019}. This creates a compounding factor for the previously mentioned disparities in home ownership. Increases in household wealth and secured debt were found to decrease the likelihood of homeowners becoming renters and vice versa \citep{anderson_effect_2021}. Money paid towards a mortgage generates long-term wealth while money paid towards rent generates wealth for the property owner. Renters with high-cost housing are unable to increase household wealth through their means of housing.  

\subsubsection{\textit{Housing Type}} 
In addition to whether one rents or owns their home, the type of home can play a significant role in housing insecurity. The housing types considered for this study are single family, multi-family, mobile/ manufactured homes, and “unconventional housing.” Unconventional housing includes dwellings not considered for long-term habitation including RVs/ campers, vans, and boats. These unconventional forms of housing may keep people off the streets, but they are not always a stable mode of housing. For RV and camper living, people who are undocumented or are unable to keep up with legal or maintenance costs for vehicles end up losing their housing \citep{wakin_not_2005}. Mobile homes also carry a unique set of circumstances that may put households at a greater risk of housing insecurity and are found frequently in rural areas. Mobile homes and the land they are situated on can be either owned or rented. It is common in mobile home parks for households to own their home but not the land it is on. Key issues with mobile homes include their financing: typically done through more expensive but easier obtained means than a mortgage such as personal property or chattel loans; mobile homes do not build wealth in the same way as they typically depreciate rather than appreciate; households on rented land have little control over their length of stay; they also tend to have worse construction and higher risks of air pollution and fire than traditional homes \citep{mactavish_wrong_2007}. 

\subsubsection{\textit{Transportation}} 
Transportation plays a large role in social and economic life. Access to everything from education to healthcare depends on the infrastructure and ability to use available means of transportation. The expense of owning enough vehicles may prove restrictive, especially for households with high housing costs. Rural areas often do not have public transportation, leading residents to depend more on automobiles. An analysis of 2009 National Household Travel Survey data found that 72\% of households with a yearly income of \$20,000 have access to a household vehicle compared to over 97\% of households making \$50,000 \citep{blumenberg_automobile_2012}. Automobile ownership can be a crucial factor in avoiding residential instability (Kang, 2019). Households are twice as likely to be auto-deficient (less than 1 car per driver) than zero-vehicle households where a vehicle is not needed \citep{blumenberg_car-deficit_2020}. This is a harrowing statistic in rural areas without public transportation where distances may be too far to walk or ride a bicycle or it may not be safe due to lacking road infrastructure like bicycle lanes and sidewalks.  

\subsubsection{\textit{Household/ aid}}
In his first State of the Union address, President Lyndon B. Johnson asked Congress to declare an “unconditional war on poverty… not only to relieve the symptom of poverty but to cure it and, above all, to prevent it.” Since then, a patchwork of programs regulated at the federal, state, and local levels have arisen. A large part of the federal government growth in the late 20th century is from the expansion of social welfare spending \citep{fishback_social_2020}. As the primary mechanism of income distribution is what \citet{berkowitz_gaps_2023} refer to as the “factor payment system” in which those who work and those who own the means of production and one’s relation to this system and the labor market is closely related to one’s poverty risk. To alleviate this poverty risk, social programs which utilize different mechanisms are available to those who qualify. These mechanisms can be divided into categorical and income targeted policy designs, alongside decentralization, where some receive benefits based on “demographically defined, categorical eligibility structures” and others enjoy standardized federal assistance through social insurance with some qualifying for income-based or “means-tested” programs \citep{bruch_poverty_2023}. Households must fall below certain income and asset thresholds to qualify for means tested programs \citep{rank_welfare_2002}. For housing, there is a wide variety of housing policies and programs aimed at low-income individuals. These take the shape of voucher programs by subsidizing privately held property although some recipients live in public housing \citep{kim_housing_2017}. For rural areas, the U.S. Department of Agriculture (USDA) has a variety of programs aimed at improving living conditions in rural areas including direct or guaranteed loans for single or multi-family housing, and infrastructure programs for water, electricity, and telecommunications ((USDA Rural Development Summary of Programs, 2023). 

\subsection{\textit{Summary}}
\endinput