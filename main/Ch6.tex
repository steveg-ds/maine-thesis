\chapter{Conclusion}	%Chapter title

This chapter presents a synthesis of the work discussed in this thesis on "Rurality and Robustness: Rural Communities and Housing Insecurity Risk". A summary of the preceding chapters highlights how the analysis of rural housing insecurity was systematically developed. Section 6.2 presents the three major conclusions related to the housing insecurity risk assignment system developed, the connectivity of housing insecurity risk, and future research possibilities. 

\section{\textit{Summary}}

The goal of this thesis is to establish a baseline for further study into rural housing insecurity. Scholars must validate this exploration's findings before we can begin to understand rural housing insecurity and homelessness. The exploratory analysis was performed using supervised and unsupervised machine learning, and spatial analysis techniques. This is an expansion of previous research into rural areas that has used similar threshold measurements to divide places into prosperous or high-poverty places. Rural areas vary greatly between and within themselves, so a methodology that can address housing insecurity in a relative rather than absolute way is necessary and important to give rural areas the attention they deserve. 

Chapter 2 builds up the 4 Cs of housing insecurity framework, describes the nuance of housing insecurity as an alternative to the housed and unhoused binary, presents some of the challenges faced by rural areas, and details the development and fundamentals of the applied methods. It demonstrates the significant overlap between each pillar of housing insecurity. Each pillar of housing insecurity reflects a vital and interrelated part of housing insecurity. Housing costs need to be affordable to have consistency. Housing conditions play a vital role in health and households may be driven to housing with worse conditions by residential instability. The context in which one lives defines much of their opportunity in life and has a major impact on their housing insecurity risk. As the rural challenges section demonstrates, rural areas are far more than simply "not-urban" as they are often defined. These include problems with poverty, geographic isolation, a growing economic divide between urban and rural places, and racial segregation. Rural areas encompass a wide range of people, social structures, and communities. Necessary to any study of housing insecurity or homelessness is the need to acknowledge the historical factors influencing the struggles of people, especially marginalized populations, to understand the underlying processes. 

Chapter 3 presents a method to apply the 4 Cs of housing insecurity to rural areas using machine learning and spatial analysis techniques. First is generating risk levels relative to other census tracts on a state-by-state basis and then analyzing those clusters to identify the clusters with a high, medium, and low risk of housing insecurity relative to each other. Association rule learning is then used to analyze common relationships between risk levels and pillars of the 4 Cs. Then, local and global Moran's \textit{I} are used to improve our understanding of the spatial nature of housing insecurity in rural areas. Finally, multinomial logistic regression is used to discover the predictability of risk levels. 

Chapter 4 presents and examines the results of the risk level assignment system. 12 percent of census tracts are identified as having a medium or high risk of housing insecurity based on a threshold of risk levels. Each census tract has a maximum risk of 8 and a minimum risk of 24. They are considered a high risk if their total is less than or equal to 12 and a medium risk if it is between 12 and 15. All other census tracts are considered at a low risk of housing insecurity. These thresholds were adopted to highlight the areas with the most urgent housing insecurity levels. The most important insights come from the cluster analysis and spatial autocorrelation results. The cluster analysis highlighted segregation in rural areas, the importance of employment in education, health, and social work in rural areas, a notable presence of high-cost renters, and that there is a concerning number of individuals who did not move but either did not have a high school diploma or are below the poverty level. The association rules show that there is an average probability of about 30 percent that a census tract with a high risk in one sector has a high risk in another sector. The global Moran's \textit{I} analysis shows that spatial autocorrelations across variables are generally low at the state and national level with outliers that future work should investigate. 

Chapter 5 synthesizes the results of the analysis with the literature, highlights areas where the results align with the literature, and points out interesting observations that future research should consider. Rural areas may share some of the same issues as urban areas when it comes to rural housing insecurity. One example of this is a significant number of rent-burdened renters, another is the high levels of racial segregation especially between Whites and African Americans. The association rules show that while there are about 10 percent of census tracts with a high-risk level in at least 2 sectors, there are a similar number of inverse high-risk-to-low-risk and low-risk-to-high-risk relations. The spatial analysis reveals that unlike what is generally expected in rural areas, the variables considered here had generally low spatial autocorrelations except for the 7 variables: the white population, American Indian and Native Alaskan, the catch-all agriculture, forestry, fishing, hunting, and mining variable, owners of mobile homes, individuals living in the same house with less than a high school education, owners of single-unit homes and the "other" demographic variable. While the variables themselves were not highly spatially autocorrelated, local Moran's \textit{I} reveals that there is notable spatial autocorrelation of high and medium-risk census tracts. The multinomial logistic regression shows that rural areas are highly unpredictable, with both national and state-by-state models performing poorly. 


\section{\textit{Conclusions}}

Housing insecurity is difficult in several different ways. First, it is difficult to define. Until there is a greater understanding of housing insecurity, which includes amending the gap between urban and rural housing insecurity research, we are limited in our ability to properly operationalize the meaning of the phrase. Second, it is difficult to study. As a concept that spans such a wide range of individual, social, and political factors is inherently difficult to study. Third, and most importantly, housing insecurity is difficult for those who experience it. In rural areas, these difficulties are compounded due to the urban-centric lens of housing insecurity that has developed over decades of primarily urban-oriented research. This thesis helps to bridge this gap with its three major conclusions. First, it identifies 280 census tracts that show significant signs of housing insecurity and 1,692 census tracts that show a medium risk of housing insecurity for researchers and policymakers to address. Second, it serves as a starting point for researchers to investigate housing insecurity at a smaller level to refine and improve this methodology. Third and most importantly, it provides significant evidence that housing insecurity may not experience the same clustering in rural areas as it does in urban areas. This initial exploration hopes to serve as a starting point for policymakers and researchers to begin deconstructing the urban-centric lens and give those in rural populations the attention and resources they need and deserve. 

\section{\textit{Future Research}}

Future research should use this study as a starting point for giving housing insecurity and homelessness in rural areas adequate attention. The most important direction is to identify community-level risk factors unique to rural areas. Further studies should also use a wider range of data sources to capture sectors with few available variables such as housing conditions. Subsequent investigations should examine rural housing insecurity at a localized level. This will enable the refinement and enhancement of this model, providing more precise insights into the unique challenges faced by rural communities.	Future endeavors should prioritize a closer examination of areas exhibiting unexpected high-risk-to-low-risk and low-risk-to-high-risk relationships, as identified through the association rules analysis. Understanding the underlying factors contributing to these unexpected relationships is essential for targeted interventions and policy recommendations. There is a need for in-depth research to discern how levels and trends in income inequality differ between urban and rural areas, shedding light on the specific socio-economic dynamics impacting housing insecurity in each setting. Future research should scrutinize the distinctions in poverty and housing cost dynamics between rural and urban areas, aiming to gain a deeper understanding of the factors at play in each context. The states highlighted in the Moran's \textit{I} outlier section warrant attention because they exhibit notably higher levels of spatial clustering of risk factors than other states. People in these states may be at a higher risk of housing insecurity relative to other states. By addressing these research gaps, researchers can better inform evidence-based policies and interventions that mitigate housing insecurity and advance the well-being of rural populations. Researchers should also experiment with applying different clustering algorithms such as hierarchical clustering which would allow for the number of risk levels derived from the data rather than explicitly chosen. By addressing these research gaps, researchers can better inform evidence-based policies and interventions that mitigate housing insecurity and advance the well-being of rural populations. As we strive to enhance housing security and social equity in both rural and urban landscapes, interdisciplinary collaboration and persistent research efforts will remain pivotal in driving meaningful societal change. 

\endinput