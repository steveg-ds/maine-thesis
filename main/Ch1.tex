\chapter{Introduction}  

Homelessness research has undergone a significant transformation in recent years. Historically, the focus was on categorizing and describing different segments of the homeless population \citep{lee_homelessness_2021}. Much of this research focused on individual risk factors rather than community-level risk factors. A contemporary approach views homelessness as a spectrum rather than a binary condition (e.g., Cox et al., 2019; Desmond et al., 2015; Swope and Hernández, 2019). This shift opens the door to a new approach to study housing insecurity. In data science terms, the housed/unhoused binary resembles logistic regression, where households are categorized as housed or unhoused. However, housing insecurity is far more complex, influenced by a multitude of interconnected factors similar to a neural netework with many inputs, outputs, and hidden layers inbetween. Data science offers promise in bridging gaps in our understanding of housing insecurity dynamics. By harnessing the power of data science, we can develop more equitable, evidence-based solutions that address the root causes of housing insecurity and improve outcomes for vulnerable populations.

Four critical areas remain unaddressed in the literature that limit our understanding of housing insecurity. First, housing research has concentrated on urban settings resulting in an urban-centric view of social issues like poverty and homelessness \citep{gleason_using_2021}. Figure~\ref{fig:urban_homeless}  shows a stereotypical view of the urban unhoused population. Drawing comparisons to Hoovervilles of the great depression, the growing presence of tent cities in urban areas in the aftermath of the Great Recession became a notable social issue \citep{herring_roots_2015}. An often under-acknowledged problem is the hidden homeless. One study conducted in Los Angeles estimated that at least 20 percent of the homeless population was "hidden", meaning they were not counted based on the criteria set by the Greater Los Angeles Homeless Count \citep{agans_enumerating_2014}.  \textcolor{white}{\citep{molina_notitle_2023}} % this is how i'm citing the photo so it will show up in the references

\begin{figure}[htbp] 

    \centering 
    -
    \includegraphics[width=\textwidth, height=8cm]{plots/homeless.jpeg} 
    
    \caption[Stereotype of Urban Homelessness]{Stereotype of Urban Homelessness (Molina, 2023)} 
    
    \label{fig:urban_homeless} 
    
\end{figure} 

\pagebreak

Those who do not meet our stereotypes of being unhoused (i.e. those sheltering in abandoned buildings, couch surfing, or living in a vehicle) are excluded from the conversation even in urban areas. The rural homeless population often remains unseen, not just because rural areas are sparsely populated, but also due to the absence of support services and amenities typically available in urban areas \citep{cloke_handbook_2006}. This is why the study of housing insecurity is a vital resource for rural areas. Without appropriate infrastructure to deal with the unhoused, preventing literal homelessness should be a priority of researchers and policymakers. Second, measuring housing insecurity is challenging because of its dependence on circumstances and obstacles for both individuals and communities \citep{leifheit_building_2022}. Third, housing and homelessness necessitate a multi-disciplinary approach to properly capture the aspects that contribute to them, an approach rarely used in extant literature. Finally, the scarcity of identified community-level risk factors in rural areas coupled with a dearth of rural-specific data and research limits our understanding of housing insecurity and rural homelessness \citep{gleason_using_2021}. Studies on homelessness often focus on descriptive surveys of those accessing public services and providers of public services \citep{robertson_rural_2007}. Addressing these gaps by integrating rural areas into the discourse on homelessness and housing insecurity is essential for creating a just and equitable society with effective policies for preventing and addressing homelessness \citep{oregan_how_2021}. 

 

\section{\textit{Rural Areas}} 

Rural areas encompass a broad spectrum of places, including farms, ranches, villages, forested areas, small towns, and many other characteristics \citep{cromartie_defining_2008}. Rural areas dominate the land mass of the United States, but with 85 percent of the population living in urban areas, they are often overlooked in the public discussion \citep{pendall_future_2016}. \citet{castle_conceptual_1998} identified a sparse population, interdependence with urban and global systems, and enormous diversity between rural communities as three general characteristics of rural places. At their core, rural areas are a function of "space, distance, and relative population density" \citep{castle_place_2011}. \citet{shoup_principles_2010} group urban areas into three categories: rural areas dependent on nearby urban centers, "destination counties" with natural or artificial amenities that attract temporary residents, and production communities that revolve primarily around a single industry. This variation makes defining and understanding rurality a difficult challenge.  Despite this variation in rural areas, "rural" is often defined as "not urban" \citep{national_coalition_for_the_homeless_rural_2009}. In the study of housing, rural areas are often excluded from the conversation \citep{gkartzios_housing_2017}. Contributing to this problem is a wide variety of definitions of rurality used by governmental organizations, policymakers, and scholars (\citealp{cromartie_defining_2008}; \citealp{yousey_defining_2018}). Recently, the main policy objective for rural communities has been the promotion of economic development and preservation of the characteristics ascribed to rural areas \citep{lichter_changing_2007}.  

Rural people are distributed over a blend of the rural-urban continuum. They make up about 20 percent of the nation's population, 13 percent of the metropolitan population, 48 percent of the micropolitan population, and 75 percent of the noncore-base area population \citep{isserman_national_2005}. As Figure~\ref{fig:pop_map} demonstrates, areas ranging from small towns to the most rural areas encompass a large mass of land with relatively few people. Deconstructing the urban-centric lens of housing research necessitates a novel approach that can accommodate the differences in rural areas. The size and variation of rural areas necessitate addressing rural issues differently because there can be no one-size-fits-all policy approach to improving conditions for rural people.  

 

\begin{figure}[htbp] 

\centering 

\includegraphics[width=1\textwidth, height=10cm]{plots/pop_map.png} 

\caption{Variation in the Rural Population} 

\label{fig:pop_map} 

\end{figure} 

\section{\textit{Measures of Homelessness}} 

Scholars have long debated if research should focus on the reasons why people become homeless or on the structural forces that create homelessness \citep{shlay_social_2003}. Much research on homelessness focused on identifying and describing categories of homeless people \citep{lee_homelessness_2021}. Researchers have given significant attention to the binary of individuals and families being housed or unhoused and trying to assign them into umbrella categories. This neglects the wide range of individual and societal factors that occur in the phases between when a household is housed and becomes unhoused. For measuring homelessness, the most popular mechanism in the United States is the Department of Housing and Urban Development (HUD) point-in-time (PIT) count and housing inventory count. These counts are used for the distribution of federal funds for combating homelessness. As \citet{agans_enumerating_2014} noted, the housed may quickly become unhoused, making it difficult to accurately estimate the number of unhoused people at any given time and the unhoused frequently relocate. When it comes to addressing literal homelessness, public health experts differentiate between preventative services and reactive or emergency services \citep{oregan_how_2021}. Preventive services prevent households from becoming homeless, while reactive or emergency services step in after a household becomes homeless. A common reactive program is a treatment program where an unhoused person is required to participate in short-term residential programs before being placed in more permanent housing \citep{evans_reducing_2019}. As homelessness is often seen as an urban problem, most intervention occurs in urban areas \citep{gleason_using_2021}. Significant federal action on homelessness began with the passage of the McKinney-Vento Homeless Assistance Act of 1987. It provided funds to support a variety of programs \citep{evans_reducing_2019}. The Homeless Emergency Assistance and Rapid Transition to Housing Act of 2009 expanded the definitions of homelessness for supported federal programs to expand those eligible beyond the literal homeless \citep{berg_hearth_2013}. These included those living in a place that is not meant for habitation, people who are expected to lose their residence within 14 days, families with children that are unstably housed, and people fleeing domestic violence \citep{evans_reducing_2019}. Federal action has significantly impacted the help available for households in need, the piece that is missing is a way to target rural areas in need of assistance. 

 

 

\section{\textit{Housing as Health}} 

 

A house is far more than four walls, a roof, and some doors, the characteristics and location of a house have a significant impact on one’s life. In the United States, housing is often a family’s greatest expenditure, their greatest source of wealth, and a place of safety and gathering \citep{braveman_housing_2011}. The federal government has long acknowledged this through legislation like the Housing Act of 1949, and social programs and development goals developed by HUD. Housing is often one of the most fundamental determinants of health and a lack of adequate housing can produce adverse health outcomes and acts as a foundation for “social, psychological, and cultural well-being” (\citealp{dalessandro_housing_2020}; \citealp{leifheit_building_2022}). A health disparity or health inequity is a difference in health or health outcomes relevant to social, political, and economic factors \citep{lutfiyya_rurality_2012}. One major factor that has been linked to health disparities is income, and this relationship exists across a wide range of socioeconomic factors \citep{canto_rural_2014}. Part of acknowledging housing as health is moving beyond the housed and unhoused binary in order to better understand and intervene in households that are at risk of becoming unhoused. This is often referred to as housing insecurity, a broader term that encompasses a continuum that affects a larger part of the population than being housed or unhoused \citep{deluca_housing_2022}. 

 

\section{\textit{Theoretical Framework}} 

 

Housing insecurity has a variety of definitions across government organizations, but its domains can be characterized as encompassing housing stability, housing affordability, housing quality and safety, and neighborhood quality and safety \citep{cox_road_2019}. Housing insecurity arises when households face the extreme negatives of these concepts.To further refine these broad characteristics, this thesis follows the 4 Cs approach to housing insecurity. With little infrastructure for homelessness services in rural areas, the 4 Cs approach to housing insecurity proposed by \citet{hernandez_housing_2019} can highlight areas of critical concern for devoting resources to reactive services and identify areas where preventative services can improve or expand. The pillars of the 4 Cs include conditions: the quality of housing, cost: the affordability of housing, consistency: residential stability, and context: neighborhood opportunity. The 4 Cs of housing are an interconnected web of factors that impact health and encapsulate the “unequal distribution of housing disparities along other axes of inequality, and the historical forces shaping unequal housing opportunities” \citep[1]{hernandez_housing_2019}. \citet{metzger_fair_2017} proposed a similar framework that encompasses stability, affordability, internal housing conditions, and area characteristics. That multiple scholars have conceptualized a similar approach indicates that it may appropriately encapsulate housing insecurity within the limitations of our understanding.

 

\section{\textit{Motivation}} 

Three primary reasons motivate this thesis. The first motivation stems from the lack of attention scholars have paid to rural areas as it pertains to housing insecurity. While the literature on rural housing insecurity is growing, there has yet to be a holistic nationwide survey of rural housing insecurity. Rural areas deserve more attention, and this thesis hopes to serve as a starting point for future research on rural housing insecurity with the ultimate goal of breaking the urban-focused lens of housing insecurity by leveraging publicly available data sources intended to study social problems. The second motivation is to provide an algorithmic implementation of the 4 Cs model of housing insecurity in the rural United States. Given the urban lens to housing insecurity research, an adequate theoretical model must be capable of adapting to areas left out of the conversation. One study has applied the 4 Cs model to housing insecurity in the state of Maine \citep{gleason_using_2021}. Most applications of the 4 Cs have been to study the relationships between various health conditions and housing insecurity, but no studies have applied it broadly to rural housing insecurity. The final motivation is to provide policymakers and researchers with a framework to identify rural areas of housing insecurity in their constituency and create harm reduction approaches and services that can meet the unique needs of their areas. The patchwork of local, state, and federal systems that encompass the aid programs of the United States means many people are involved in the policy-making process with no adequate mechanism for addressing housing insecurity in their constituency. The field of data science has a variety of techniques such as clustering and classification that can assist in targeting aid towards communities that have an elevated risk of housing insecurity.

 

\section{\textit{Approach}} 

 
In order to improve our understanding of rural housing insecurity, this thesis investigates the risk levels of rural census tracts in the United States under the 4 Cs model of housing insecurity from the data science perspective. Data science is a multidisciplinary field that employs scientific methods, algorithms, and systems to extract insights and knowledge from structured and unstructured data. It integrates expertise from statistics, mathematics, computer science, and domain-specific knowledge to analyze and interpret complex data sets. The goal is to enable informed decision-making and uncover valuable patterns within data sets.Risk factors across eight different axes are used to assign risk levels to rural census tracts. Each state is clustered with census tracts from other states within a 15-mile boundary to encapsulate how communities span across state lines. The clusters are analyzed to understand the trends in housing insecurity factors across states and the clusters are relabeled based on risk factors identified in the literature so that each cluster represents a low, medium, or high risk of housing insecurity. These risk levels are used to highlight census tracts at a low, medium, and high risk of housing insecurity relative to other census tracts in their state. Association rules learning is used to identify common patterns between sector risk levels. To better understand how factors of housing insecurity relate to space, local and global Moran's \textit{I} spatial autocorrelation is used to determine how spatially clustered each housing insecurity factor is. Local Moran's \textit{I} is used to determine how spatially clustered each risk level is to better understand the spatial clustering of housing insecurity risk in rural areas. Finally, a multinomial logistic regression is used to determine how well each state's sector risk levels can be predicted, and a national model is generated for each sector's risk levels to analyze how well the risk levels created by this implementation of the 4 Cs model can be predicted nationally and state by state. Beginning to understand rural homelessness requires that several questions be answered:
    \begin{itemize}
      \item How can housing insecurity risk factors be used to identify risk levels of housing insecurity while accounting for the variation in rural areas and what do the risk levels say about rural areas?
      \item When measuring housing insecurity across different dimensions, how often do the same features arise?
      \item What are the spatial relations between the different dimensions of housing insecurity?
      \item To what extent can this model be used to predict housing insecurity risk levels across identified factors?
    \end{itemize}

  
\section{\textit{Major Results}} 

The work presented in this thesis presents a novel application of the 4 Cs of housing insecurity framework and applies it to rural areas. This framework allows for the identification of 280 census tracts with a high risk of housing insecurity and 1,692 census tracts with a medium risk of housing insecurity relative to other census tracts in their state and neighboring communities. The association rules learning results show that there are a notable number of unexpected relationships where a high risk in one sector is associated with a low risk in another sector and vice versa. It presents evidence that clustering of housing insecurity factors may not be as common in rural areas as they are in urban areas. It also shows that it is difficult to predict the risk levels of census tracts with state and national models. Due to its exploratory nature, the results are primarily intended to be used as a starting point for future research into rural housing insecurity.  

 

\section{\textit{Intended Audience}} 

This thesis is intended for an audience with a significant interest in rural housing insecurity. Such an audience can include policymakers, economists, political scientists, community psychologists, rural sociologists, social data scientists, geographers, demographers, and many others concerned with housing insecurity and rural areas.  

 

\section{\textit{Structure of Thesis}} 

The thesis is structured into six chapters. Chapter 2 offers a comprehensive theoretical foundation, focusing on the application of the 4 Cs of housing insecurity model and the study of data science as it relates to the methods used and previous applications of data science methods to housing insecurity-related factors. This chapter reviews pertinent literature on various facets of the model. Chapter 3 explains the methodology employed for data processing and analysis. It provides an in-depth explanation of the methodology for applying the theoretical framework and its execution. Chapter 4 presents the study's findings, offering a detailed analysis of the acquired results. Chapter 5 deliberates on the results, discussing their significance and impact within the scope of the study. This chapter provides a thorough examination of the noteworthy findings. Chapter 6 serves as a synthesis, summarizing the entirety of the work and offering insightful commentary on the major findings. Additionally, it highlights potential avenues for future research and study. 

 

\endinput 

 