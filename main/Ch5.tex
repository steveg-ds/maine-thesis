\chapter{Discussion}	

\hl{How can risk factors of be used to identify risk levels of housing insecurity while accounting for the variation in rural areas?}

The analysis of this thesis used the 4 C's of housing insecurity theoretical framework with American Community Survey data to identify risk levels of housing insecurity in a way that accounts for the variation in rural communities. The cluster analysis highlights the variation that exists in rural areas.

\hl{Do housing insecurity factors identified for urban areas exhibit similar characteristics in rural areas?}

\hl{When measuring housing insecurity across different dimensions, how often do the same features arise?}

\hl{Are there spatial relations between the different dimensions of housing insecurity?}

\hl{To what extent can the housing insecurity dimensions be predicted?}



\section{\textit{Employment Diversity}}

There are \hl{?} observations to be made from the employment diversity cluster analysis. First, at over 9 percent, the data shows that employment in education, health, and social work contributes significantly to employment in rural areas. The cluster medians for all three clusters are almost double the next highest values in retail trade and manufacturing. Next, wholesale trade and information employment make up relatively small parts of the rural economy with less than 1 percent across all cluster medians. The cluster medians for the catch-all "rural jobs" variable, incorporating agriculture, forestry, fishing, hunting, and mining, falls in the middle in terms of contributing to rural economies. This highlights the shift from agricultural to manufacturing that rural economies have made over the last several decades (\hl{source}). 

\section{\textit{Demographics}}
There are \hl{?} observations to be made from the demographic diversity cluster analysis. First, the data indicate that rural areas are predominantly white, with cluster medians ranging from 92 to 94 percent white. However, as previous research has indicated (\hl{source}), there are pockets of large minority populations and this is captured by the cluster analysis as demonstrated by the cluster with the lowest white population having the highest African American, Hispanic and Latino, and Other populations. The age and gender variables reflect a similar distribution of men and women over and under 18. As found by (\hl{source}), the data indicate that rural areas have a much larger older population than younger population with cluster medians falling between 9 and 11 percent for both under-18 variables. 


\section{\textit{Housing Cost}}
The cluster medians of the housing cost sector indicate that rural areas show a similar trend in housing costs. The cluster medians for the high-cost renters variable are almost three times as high as the next highest cluster medians, for the mortgage high-cost variable. One important factor in rural areas that has not been taken into account for housing costs is the widespread presence of official or unofficial mobile home parks. The effects of these structures on rural housing costs require further research to be fully understood. Another notable observation is that the cluster with the most high-cost renters also has the most high-cost homeowners without a mortgage. This indicates there may be some clustering of the different categories of housing cost.


\section{\textit{Housing Quality}}

The cluster medians of the housing quality section indicate that the ratio of occupied to unoccupied housing with incomplete plumbing or kitchen facilities is significant. The cluster medians for the incomplete facilities variables range from 17 to 25 percent while the medians for the occupied housing variables are never greater than 0.64 percent. This sector is severely limited because these were the only ACS variables that capture housing conditions. More factors are needed to fully encapsulate housing conditions, but these variables can at least highlight areas that show signs of poor housing conditions. 


\section{\textit{Residential Mobility: Education}}
Rates of residential mobility for the education sector were low, the highest cluster median is the variable for those who moved in the same county with a high school degree at 1.36 percent. The cluster medians for the same house with a high school degree variable ranges from 22 to 23 percent, making the category with the most stable situation far greater than those who moved even when all transiency variables are combined. One note of concern is in all three clusters, 7 percent of the population did not move, but do not have a high school diploma. While stable for the year for which they were interviewed, their education status puts them at risk for housing instability. 

\section{\textit{Residential Mobility: Poverty}}
There are \hl{?} observations to be made from the residential mobility: poverty cluster analysis. First, the cluster medians for same house below the poverty line are higher than the same house at 125 percent of the poverty level variable. This aligns with the rural pockets of poverty indicated by (\hl{source}). Next, Across cluster medians, the percent of the population in poverty or just above the poverty line is never greater than 1 percent. Finally, For transiency variables in this sector, moved in county below the poverty level has the highest cluster medians across all variables indicating that this group is the most present in the dataset. 

\section{\textit{Household Factors}}
There are \hl{?} observations to be made from the household factors cluster analysis. First, there is a significant conversation about income inequality in urban areas, yet the cluster medians for the Gini index are less than 0.5 indicating a low level of income inequality in rural areas. Next,  for each cluster median, at least a third of the population has no investment income or no other income, indicating that residents of rural areas are highly dependent on wages. The cluster medians for the number of households receiving public assistance range from four to five percent, indicating there may be a significant number of households eligible for public assistance that do not have it. The cluster medians for the number of households with more than three workers are less than two percent, indicating that this high-risk factor occupies a minority of households. 


\section{\textit{Housing Type}}

There are \hl{?} observations to be made from the housing type cluster analysis. First, single-unit homeowners are the primary means of housing in rural areas, with cluster medians ranging between 88 and 90 percent. The next most prominent type of housing is single-unit renters, with cluster medians ranging from 55 to 68 percent. This is significant given the previously mentioned levels of high-cost renters in rural areas. Numerous studies (\hl{insert sources}) have identified the harms of high-cost renting, but the extent to which these effects are the same in rural areas has not been identified. 


\section{\textit{Association Rules}}

The association rules show that the risk levels of each sector do not have a lot of common overlap. Of the four types of associations analyzed, there were no associations that occur more than expected by random chance. This means that rural areas, when space is not accounted for, have little commonality in terms of housing insecurity risk across the different sectors. This is unexpected under the 4 C's of housing insecurity framework where areas theoretically show similar signs across pillars. While the association rules did not find strong trends between risk levels, it does highlight pockets of rural census tracts where this commonality exists. These areas may be of interest to policy-makers and researchers for their communities. 

\section{\textit{Moran's I}}

When spatial relationships are accounted for, the results are very similar to the association rules. Of the 2,018 statistically significant Moran's I values, the average of 0.26 indicates that there are generally low spatial autocorrelations between variables from a wide lens. The sector averages tell a very similar story, ranging between 0.2 and 0.32 for each sector. These results indicate that rural areas do not have the same level of clustering as noted with many of the housing insecurity factors. The outliers highlighted in Chapter 4 point to areas of concern that further research should investigate. While the global Moran's I values do not indicate very high levels of spatial autocorrelation among factors, the local Moran's I analysis reveals that there are strong local spatial autocorrelations between high-risk and low-risk level census tracts across each sector. While these numbers are influenced by the way that census tracts were only clustered with relatively close census tracts, it provides some evidence for the clustering of housing insecurity risk levels in rural areas. Most notable is the level of spatial randomness for the medium-risk levels across sectors. The results indicate that the extreme clusters are spatially clustered but not in the middle area. 


\section{\textit{Multinomial Logistic Regression}}

The multinomial logistic regression results show that it is very difficult to predict the housing insecurity risk levels of census tracts based on other states and nationally. This provides evidence to one of the themes of rural research which emphasizes that rural areas vary greatly. The models also faced significant issues with over-classifying census tracts as low-risk. This could indicate that there is less variability in these census tracts, but this is not beneficial in terms of understanding rural housing insecurity. 

\section{Evaluating the 4 C's}

While this exploratory analysis has provided novel insights into rural housing insecurity and how the current scope of housing insecurity research can be adapted to rural areas, the results did not closely follow what is expected under the 4 C's of housing insecurity framework. There are three plausible explanations for this difference between the theoretical model and the results. First, the methodology used in this thesis does not adequately encapsulate the 4 C's of housing insecurity. Second, the urban lens of housing insecurity does not apply to rural areas enough for the same effects from the housing insecurity factors. Third, the 4 C's model of housing insecurity, potentially biased by the urban lens of housing insecurity, does not conform well to rural areas. 

\section{\textit{Previous Research}}

Due to the urban-centric lens towards housing insecurity, there is little previous research to compare this study to. Gleason et al. 2021 applied similar spatial techniques to census tracts in Maine found that poverty, unemployment, and high housing costs are common in rural and urban areas of Maine but found these results to be inaccurate in a later study (Gleason et al 2022?). Lichter and Johnson 2007 did a nationwide county-level analysis on poverty levels specific to rural areas. Insert authors that did research specific to rural areas. This is the first study of this scale to employ data mining techniques on rural housing insecurity. Also, it develops a methodology used with the 4 Cs of housing insecurity that can be applied to rural and urban areas while moving the focus further away from homelessness as a binary. 


\section{\textit{Limitations}}
As the American Community Survey samples less in rural areas, the accuracy of the data is limited in how accurately it represents the real-world. While the estimates are “likely reasonable approximations of the populations they represent”, small area estimates like census tracts used here have issues with attribute uncertainty (Spielman 2014, p#). Despite this, it is currently the most detailed source of data available for rural areas. The largest issue that needs to be addressed in rural housing insecurity research is the lack of factors uniquely identified to rural areas. This study relied on rural-specific literature as much as possible, but many of the works cited are from an urban lens. Future research should use this study as a starting point for giving housing insecurity and homelessness adequate attention. The most important direction is to identify community level risk factors unique to rural areas. Further studies should also use a wider range of data sources to capture sectors with few available variables such as housing conditions. From our spatial analysis there are several observations worthy of further investigation  