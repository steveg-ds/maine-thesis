\chapter{Introduction} 

\subsection{\textit{Background}}
Homelessness research has undergone a significant transformation in recent years. Traditionally, the focus was on categorizing and describing different segments of the homeless population \citep{lee_homelessness_2021}. A more contemporary approach views homelessness as a spectrum rather than a binary condition (e.g., Desmond et al., 2015; Swope and Hernandez, 2019). This paradigm shift opens opportunities for preventative and reactive services to address homelessness and housing insecurity. To harness these opportunities, three critical areas in literature require attention. First, housing research has concentrated on urban settings resulting in an urban-centric view of social issues like poverty and homelessness. Second, measuring housing insecurity proves challenging due to its dependence on circumstances and obstacles for individuals and communities \citep{leifheit_building_2022}. Rural areas encompass a broad spectrum, including farms, ranches, villages, small towns, and many other characteristics \citep{cromartie_defining_2008}. Finally, the scarcity of identified community-level risk factors in rural areas, coupled with a dearth of rural-specific data and research, makes the study of rural homelessness particularly intricate 
\citep{gleason_using_2021}. Addressing these gaps by integrating rural areas into the discourse on homelessness and housing insecurity is essential for creating a just and equitable society with effective policies for preventing and addressing homelessness \citep{oregan_how_2021-1}.

\subsecion{\textit{Rural Areas}}

% insert research questions here 
\subsection{\textit{Motivation}}
I have a little bit

\subsection{\textit{Approach}}
I have one

\subsection{\textit{Major Results}}
I have some

\subsection{\textit{Intended Audience}}
Why would anyone read this 

\subsection{\textit{Structure of Thesis}}
It doesn't have any yet
\endinput