\chapter{Addressing Rural Housing Insecurity}	%Chapter title
Eight different sectors of 2019 ACS 5-year variables are used to capture the 4 Cs of housing insecurity using indicators of housing insecurity identified in the literature. These sectors are housing cost, housing quality, housing type, economic diversity, education mobility, poverty mobility, and household worker/aid. Machine learning algorithms were used to assign each census tract with a USDA Rural-Urban Continuum code seven or greater with a label between one and three where one indicates the highest signs of housing insecurity and three indicates the lowest signs of housing insecurity based on risk indicators identified in the academic literature. 

\textit{Table 1 about here}

The range of RUCA codes (see Table 1) was chosen to be more inclusive than exclusive, including small towns with various levels of commuting to urban clusters and areas classified as rural. We include small towns because they often serve as hubs for rural areas, serving an important role in rural areas. Spatial autocorrelation is used to determine how often similar rates of each variable occurred across each rural census tract in each state. Finally, multinomial logistic regression is used to determine how well the risk levels of a census tract can be predicted based on the nationwide dataset.  All analysis was conducted in the R statistical language.  

\subsection{\textit{Data processing}}
In order to ensure the integrity of the data, census tracts that lacked specific sector-related information were deliberately excluded from the analysis. These omitted tracts were assigned a risk level of zero, a measure adopted to preserve the largest possible number of census tracts for subsequent analyses. To mitigate potential biases stemming from differences in population sizes and geographic areas, a standardized approach was employed. This involved scaling all dataset components to a common base unit. Notably, demographic and economic diversity metrics were adjusted proportionally to the population size. Meanwhile, data pertaining to household expenses and types were scaled based on the respective counts of homeowners and renters. The household dataset underwent normalization corresponding to the total number of households, whereas housing condition indicators were adjusted relative to the total count of occupied and unoccupied housing units. It's essential to note that all numerical values within the dataset have been uniformly represented as percentages, except for the household Gini Index, which retains its original values.

For demographic variables we use seven variables including an “other” variable to account for race/ ethnicity and the number of people over or under 18 by gender. The economic diversity data is the number of people employed across 13 distinct categories. There are It was necessary to create three compound variables: high-cost with a mortgage, high cost without a mortgage, and high-cost rent to use the standard affordability measure of 30 percent. There are four variables for housing conditions which include houses with an incomplete or insufficient kitchen or plumbing for occupied and unoccupied housing units. Two sets of variables account for residential mobility: education and poverty that include those who did not move, those who moved in and out of county and state.  Due to the housing affordability and income inequality crises, those below the poverty level and those at 125 percent of the poverty level as high risk for housing instability are included in residential mobility: poverty. For education those with and without a high school diploma are included as those without a college degree may face higher barriers to well-paying and stable employment. Wage/aid data include households without income, households that receive public assistance, households that receive supplemental security income, households with investment income, households with other income, households with 3+ workers and the household Gini index. For housing type, renters and owners of mobile homes, single family homes, small and large multi-family homes, and renters and owners of unconventional housing are included.  
\endinput