\chapter{Discussion}	

RQ1 Across all sectors except employment diversity, the smallest cluster count is the number of high-risk census tracts. In all sectors except housing cost and demographics, there are more low risk-level census tracts than medium-risk census tracts. Demographics has a much lower number of high-risk clusters proportional to the other risk levels compared to other sectors. The high number of high-risk census tracts in the housing cost sector is concerning as the social consequences of high housing costs are well documented. This distribution pattern underscores potential areas for targeted housing insecurity interventions and policy considerations. 

RQ2 the association rules show us that while there are some pockets of high-risk-to-high-risk, there are a considerable number of high-risk-to-low-risk and low-risk-to-high-risk relationships in the dataset, contrary to what was expected after reviewing the literature on housing insecurity and related factors. The number of high-risk-to-high-risk relationships that the household wage/ aid sector has with employment, housing costs, mobility poverty, and housing type demonstrates the interconnectedness of housing insecurity However, the high-risk-to-low-risk and low-risk-to-high-risk associations highlight the variation in rural areas demonstrated by the spatial analysis. Based on the 4 C’s model, sectors should have similar risk levels. While the results reflect this, the mix of results implies that adjustments are needed to this implementation of it or the theoretical foundation of the 4 Cs model.  

RQ3 Numerous studies have shown that poverty clusters around itself in urban areas (Foulkes \& Schafft, 2010) as the spatial analysis shows, in rural areas there is little clustering of high levels of poverty for those that moved and those that did not. High housing costs also do not conform to what was expected, with low levels of spatial clustering. One significant observation is the notably higher Moran’s I statistic for high-cost housing with a mortgage. In urban areas, renters tend to have higher costs than homeowners. One area where rural and urban areas are similar is in levels of racial segregation. African Americans, Whites, Hispanics, and Latinos have some of the highest spatial autocorrelations in the dataset. The presence of pockets of Hispanics was identified in literature by Lichter (2020) and this analysis supports their findings. Economic diversity results reflect some of the stereotypes of rural areas.  Jobs considered to be more rural like mining, agriculture, forestry, and manufacturing had higher spatial autocorrelations than “urban jobs” that primarily serve a consumer rather than a producer community. The results also indicate that the ten previously mentioned states have notably stronger spatial autocorrelations than the overall average. For both residential mobility sectors, those who did not move regardless of high school diploma or poverty status have the highest spatial autocorrelations of their sectors. For all residential mobility variables there are small positive spatial autocorrelations. Desmond and various collaborators have identified patterns of residential mobility, primarily based on a sample of Milwaukee residents. These studies provide great context into residential mobility in urban areas, but no similar studies have been done unique to rural areas. Housing quality was difficult to measure as there are a limited number of variables in the ACS that relate to it. Unoccupied housing with incomplete facilities is more spatially clustered than occupied housing with incomplete facilities. This necessitates a further analysis with a broader range of housing condition features accounted for. Housing cost variables did not have strong spatial autocorrelations except in the previously mentioned states with an average of 0.5 for these states and 0.28 for the entire sector. Research has identified mobile homes as a common housing solution in rural areas, despite their potential health and financial risk. The prevalence of high spatial autocorrelation reflects the widespread use of mobile homes in rural areas. The other variables of interest in this sector are the owner and renter unconventional housing variables which have much lower spatial clustering than single unit and mobile homeowners. Regarding RQ3, there does not appear to be significant spatial clustering of the housing insecurity indicators. This indicates that the urban-centric understanding that indicators of housing insecurity tend to spatially cluster, the same may not be true for rural areas. Alternatively, it may be that urban-centric indicators of housing insecurity do not translate well to rural conditions.  

RQ4 The multinomial logistic regression reveals how different rural areas are. Figure Z shows the distribution of probabilities across sectors. While most of the sectors are close to being symmetrically distributed, there are a lot of outliers across most sectors. The same trend exists when looking at state averages. The symmetrical distribution indicates similar levels of variation across states as values are similar on each side of the mean. A more advanced classification algorithm such as K-Nearest Neighbors or a Bayesian classifier may be more accurate but based on the multinomial distribution, the answer to RQ4 is that states cannot be predicted very accurately based on the other states in the data set. The implication is another reason a greater understanding of rural-specific housing insecurity indicators is needed and that researchers should find methods of studying rural housing insecurity that can accommodate the differences within and between rural areas. 
\textit{Figure 5 about here}
The risk assessment system is represented with a map where each census tract’s color is based on a combination of its risk level and cluster probability based on the multinomial logistic regression (see figure 6). Risk levels are denoted as red for one, orange for two, and green for three. Many rural areas fall somewhere in the middle in terms of housing insecurity risk. There are pockets of high risk and low risk census tracts that can be seen in the national map or the regional breakdown. The light colors on the map indicate that the multinomial logistic regression models were not able to predict the risk level of census tracts very well. The trends reflected in the relative risk assignment system signify a wide dispersion of housing insecurity in rural areas.  

Due to the urban-centric lens towards housing insecurity, there is little previous research to compare this study to. Gleason et al. 2021 applied similar spatial techniques to census tracts in Maine and found that poverty, unemployment, and high housing costs are common in rural and urban areas of Maine but found these results to be inaccurate in a later study (Gleason et al 2022?). Lichter and Johnson 2007 did a nationwide county-level analysis on poverty levels specific to rural areas. Insert authors that did research specific to rural areas. This is the first study of this scale to employ data mining techniques on rural housing insecurity. Further, it provides researchers with something to compare to as they begin to conduct more research specific to rural areas. Policy makers at all levels of government can use this as a tool for delegating resources available to their community. One significant limitation is the lower rate of ACS sampling in rural areas, the accuracy of the data is limited in how well it represents the real-world. While the estimates are “likely reasonable approximations of the populations they represent”, small area estimates like census tracts used here have issues with attribute uncertainty (Spielman 2014, p\#). Despite this, it is currently the most detailed source of data available for rural areas.  

Future research should use this study as a starting point for giving housing insecurity and homelessness adequate attention. The most important direction is to identify community level risk factors unique to rural areas. Further studies should also use a wider range of data sources to capture sectors with few available variables such as housing conditions. From our spatial analysis there are several observations worthy of further investigation. Subsequent investigations should focus on examining rural housing insecurity at a localized level. This will enable the refinement and enhancement of the existing model, providing more precise insights into the unique challenges faced by rural communities.	Future endeavors should prioritize a closer examination of regions exhibiting unexpected high-risk-to-low-risk and low-risk-to-high-risk relationships, as identified through association rules. Understanding the underlying factors contributing to these unexpected relationships is essential for targeted interventions and policy recommendations. There is a need for in-depth research to discern how levels and trends in income inequality differ between urban and rural areas, shedding light on the specific socio-economic dynamics impacting housing insecurity in each setting. Future research should scrutinize the distinctions in poverty and housing cost dynamics between rural and urban areas, aiming to gain a deeper understanding of the factors at play in each context. By addressing these research gaps, researchers can better inform evidence-based policies and interventions that mitigate housing insecurity and advance the well-being of rural populations. As we strive to enhance housing security and social equity in both rural and urban landscapes, interdisciplinary collaboration and persistent research efforts will remain pivotal in driving meaningful societal change. We encourage scholars and practitioners to join hands in this endeavor, working collectively towards a future where safe and stable housing is a fundamental right for all.  
\endinput