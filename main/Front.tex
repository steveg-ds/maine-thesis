\copyrightpage[Steve Garcia]{2024}		% Optional, comment out or delete if undesired

\begin{abstract}
This thesis explores rural housing insecurity through Swope and Hernandez’s (2019) 4 C's of housing insecurity in rural areas. Little attention has been paid to rural areas in the conversation on housing insecurity and houselessness. To facilitate further discussion on this understudied issue, this exploratory study used unsupervised machine learning to group census tracts into risk levels across 7 sectors of data from the American Community Survey. These were based on housing insecurity factors found in the literature. Multinomial logistic regression was used to determine variation between U.S. states based on how well state risk levels could be predicted with the national dataset. Additionally, spatial autocorrelation was used to analyze how spatially clustered the risk levels and housing insecurity risk variables.  The results indicate that many rural census tracts have a medium risk of housing insecurity, and the risk levels are hard to predict. The spatial autocorrelation results show that the housing insecurity variables were not as highly spatially clustered as expected.  
\end{abstract}

\begin{layabstract}{Housing Insecurity, homelessness, data mining}	% Replace the ... with the list of keywords
This thesis explores rural housing insecurity through Swope and Hernandez’s (2019) 4 C's of housing insecurity in rural areas. Little attention has been paid to rural areas in the conversation on housing insecurity and houselessness. To facilitate further discussion on this understudied issue, this exploratory study used unsupervised machine learning to group census tracts into risk levels across 7 sectors of data from the American Community Survey. These were based on housing insecurity factors found in the literature. Multinomial logistic regression was used to determine variation between U.S. states based on how well state risk levels could be predicted with the national dataset. Additionally, spatial autocorrelation was used to analyze how spatially clustered the risk levels and housing insecurity risk variables.  The results indicate that many rural census tracts have a medium risk of housing insecurity, and the risk levels are hard to predict. The spatial autocorrelation results show that the housing insecurity variables were not as highly spatially clustered as expected. 


\end{layabstract}

\begin{dedication}
Dedicated to St. Thomas More and all who seek to build a better world.
\end{dedication}

\begin{acknowledgements}
I would like to thank the following people for helping with this research project:
Thank you to my advisor, Dr. Matt Dube for providing guidance and feedback throughout this project, To my beautiful wife Katie for encouraging me to pursue my dreams, to Drs. Megan Smith, Jeremy Mhire, and Dolliann Hurtig for guiding me through my academic journey, and to my grandparents Steve and Linda Mauldin for guiding me through life.   
\end{acknowledgements}

\pagebreak

% Commands for the required lists
\tableofcontents

\pagebreak

\listoftables	

% Include only if there are tables in the thesis
\pagebreak
\listoffigures		
\pagebreak		% Include only if there are figures in the thesis


%\listof{equations}{List of Equations}

% Sets the document spacing and pagestyle.
\mainmatter

\endinput
