\chapter{Factors of housing insecurity}	%Chapter title

\section{Challenges for Rural Areas}

 While there is limited research on homelessness in rural areas, previous research has documented the unique struggles of rural areas that should be addressed in a discussion on rural housing insecurity. First, previous research has identified both pockets of prosperity and pockets of deep poverty in rural areas. Poverty is acknowledged more in urban areas, but poverty rates are highest in both remote rural counties and in cities. persistent poverty, typically defined as poverty levels above 20 percent, is geographically concentrated in rural regions \hl{(Crandall and Weber 2004)}. In 2010, the poverty rate among the rural population ws higher than that of the nation overall \hl{(Lichter and Eason 2013)}. A cluster analysis found that of 3,017 places with about 5 percent of the nations population, 84 percent of this population lives in rural rather than urban areas \hl{(Peters 2009)}. \hl{Lichter and Johnson (2007)} found that 85 percent of the nearly 500 counties with poverty rates over 20 percent and the 12 counties with poverty rates over 40 percent are in nonmetro areas. Despite these negative findings, there are more than 300 counties spread across the nation that are more "prosperous" than the rest of the nation based on measures spanning education, housing, poverty, and unemployment \hl{(Isserman et al., 2009)}.  This highlights the need for an approach to rural areas that is relative rather than holistically. One explanation for the high rates of poverty in rural areas is their isolation. Isolation stems from limited ease of travel or access to nearby markets and population centers which can hinder economic development, meaning that greater geographic isolation is associated with both lower income and greater poverty rates \hl{(Blank 2005)}. \hl{Metzger and Khare (2017)} offer an explanation for these pockets of rich and poor places in identifying  tendency for Americans to segregate themselves not only based on race but on class too. A tendency for the rich and the poor to cluster around themselves could explain these findings in rural areas. This spatial inequality is critical to understanding rural poverty \hl{(Valasik, 2018)}

 Another problem that rural areas are facing is a growing economic divide between urban and rural areas \hl{(Bjerke and Mellander, 2019)}. One aspect of this is that a friction is created when rural households re too distance from adequate labor markets that enable them to support their families \hl{(Sparks et al., 2013)}. This has created a common migration pattern where so many people move to urban areas for greater economic opportunities that rural towns are left with a smaller, older population and a less skilled labor force \hl{(Bjerke and Mellander, 2019)}. The effects of these population decreases span across socioeconomic factors. school consolidations, reductions in local services, closed businesses, and increased infrastructure costs have all been tied to population shrinks and communities have little ability to control these processes \hl{(Zarecor et al., 2021)}. Rural communities have been hit hard by economic changes in recent decades \hl{(Pendall et al., 2016)}

\section{Housing Insecurity}
\hl{insert stuff about housing insecurity here}

\section{The 4 C's}

To construct the 4 C's framework, the literature review is divided into one section for each pillar of housing. As each column forms a web rather than separate pieces, there is a significant amount of overlap between columns. 
 
\subsection{\textit{Cost}}
It is difficult to determine one number that determines when a household is spending too much on housing. A cost-to-income ratio is the most common way of measuring housing affordability. The threshold for housing affordability has ranged between 25 and 50 percent but the current standard is 30 percent \citep{kropczynski_insights_2012}.  Housing is considered affordable if the household spends less than 30 percent of its income on housing and 50 percent or more is considered a high-cost burden (\citealp{braveman_housing_2011};\citealp{swope_housing_2020};\citealp{weicher_housing_2006}). Inherit to any cost-to-income ratio is the understanding that housing is that there are other expenses necessary for survival \citep{herbert_measuring_2018}. Housing costs are determined by the rate of household formation and household attrition \hl{Pendall et al., 2016}.  Housing affordability affects individuals, families, and communities while access is largely determined by their demographic characteristics  (\citealp{braveman_housing_2011};\hl{Yadavalli et al., 2020}).Housing affordability is directly related to residential stability and has the potential to harm both those being forced to move, the community they are leaving, and the community they are entering (Desmond et al., 2015). Access to affordable housing affects the physical and material comfort of the population as well as the individual \hl{(Kumar, 2003)}. If a household cannot afford to live in their current place, they may be forced to relocate seeking more affordable housing or through eviction and foreclosure..  If too much of a household’s money goes to housing, they may be forced to go without other necessities \citep{herbert_measuring_2018}.   note that a body of research has found that those with high housing costs may also experience food insecurity as food is often considered a flexible expense while housing is a fixed expense (\citealp{fletcher_assessing_2009};\citealp{kropczynski_insights_2012}). The shortage of affordable housing drives lower-income families to substandard housing in worse neighborhoods \citep{braveman_housing_2011}. \citet{kang_severe_2021} characterizes housing instability as a by-product of the affordable housing shortage wherein households can be destabilized by minor financial shocks. \hl{insert housing shortage stuff} These  factors can create a situation where housing costs lead to residential instability, which is linked to a variety of adverse conditions, especially in children and adolescents \citep{desmond_forced_2015}. Part of the blanket construct of rural areas is that they are cheaper to live in. However, \hl{Kurre (2003)} note that there is relatively little systematic data that supports this presumption. Rural areas face the same low per capita income and poverty problems faced by urban areas \citep{castle_place_2011}. \hl{Zimmerman (2008)} found no consistent pattern of lower princes across all of the rural counties in \hl{?} While the dollar amount paid for housing may be lower, given the different socio-economic circumstances of rural areas, housing costs alone may not fully encapsulate the situation \citep{kropczynski_insights_2012}. 

\subsection{\textit{Conditions}}

Many scholars have identified internal housing conditions as a significant factor on health (\citealp{braveman_housing_2011};\citealp{metzger_fair_2017}; \citealp{swope_housing_2020}. In one study, decent housing was found to be a more important determinant of health than education or income \hl{(Stefan and Bittschi 2014)}. Previous environmental health research has identified five broad categories in which housing conditions contribute to adverse health effects: \textit{physical conditions}, \textit{chemical conditions}, \textit{biological conditions}, \textit{building and equipment conditions}, and \textit{social conditions} \citep{jacobs_environmental_2011}. Adequate housing is necessarily related to public health \citep{matte_housing_2000}. Links to an increase of disease have been tied to poverty, poor housing, and degraded environments reflecting the interconnectedness of housing insecurity issues \citep{rauh_housing_2008}. \hl{Stefan and Bittschi (2015)} found that the probability of facing a chronic disease increases when housing problems accumulate and that poor housing conditions quickly degrade subjective health. These problems are amplified in the modern world where individuals spend an estimated 90 percent of their time in doors \hl{(Palacios et al., 2020)}. Despite housing conditions playing such a significant role in modern life, there is not a significant sense of communal benefit and responsibility when it comes to housing \citep{jacobs_environmental_2011}. Rural areas face unique housing issues because one of the most common housing solutions is mobile homes.\hl{more on mobile homes} Structural problems like poor construction and risks of air pollution and fire create a unique problem \citep{mactavish_policy_2006}. An area of particular concern are marginalized populations who are more likely to be exposed to harmful housing conditions \citep{swope_housing_2020}. Housing conditions also play a role in residential mobility as \citet{desmond_housing_2015} place decent housing and affordable housing as fundamentally connected and the previously mentioned rise in housing cost has not brought an increase in housing quality.  

\subsection{\textit{Consistency}} 

Residential mobility is a complicated subject because, as a broad concept, it is conceived as a good thing. That one can pack up and go somewhere with more opportunity is considered a part of the American “mystique” \citep{molloy_internal_2011}. An average of 15 percent of Americans move every year and 25 percent move over the course of two years \citep{bachmann_ins_2014}. Classic urban economic theories explain that households make trade-offs between proximity to jobs and housing prices \citep{hu_housing_2019}. This puts low-income households at a disadvantage as their access to jobs may be lower than their wealthier counterparts. Consistency or residential stability plays an important role in the physical and social well-being of individuals, families, and communities. It has been linked to a variety of adverse conditions and affects the neighborhoods being entered and left. It has been identified as a more important predictor of community health than standard factors like poverty and racial composition (\citealp{desmond_forced_2015};\citealp{desmond_housing_2016}, \citealp{rauh_housing_2008}). An important distinction must be made between voluntary and involuntary moves \citep{siskar_who_2019}. While most moves are voluntary, there are millions of low-income households that struggle to maintain housing stability (\citealp{phinney_exploring_2013};\citealp{kang_why_2019}). Housing is often the biggest expense for low-income families, often forcing them to make trade offs between necessities \citep{desmond_housing_2015}. Of particular concern is forced relocation, foreclosure, eviction, and condemnation are all drivers of forced relocation (\citealp{phinney_exploring_2013};\citealp{siskar_who_2019}). It is linked to an increase in residential instability and households forced to move often end up in places with greater disadvantage and are more likely to face additional moves \citep{desmond_forced_2015}. One issue with the study of residential mobility is the limited scope of predictors that have been linked to it \citep{kang_why_2019}. In urban areas, renters have been found to be particularly vulnerable to relocating to worse neighborhoods than the one they are exiting \hl{(Desmond and Shollenberger 2015)}. It is yet to be seen how this translates to rural areas, where renting is far less common than urban areas with the exception of mobile homes. \hl{insert mobile home costs}

\subsection{\textit{Context}}

Context revolves around neighborhood and community characteristics including demographics, green spaces, education, and healthcare among other things. While it is impossible to capture context in its entirety, this thesis focuses on demographics, economic diversity, housing type, and household wage/ aid factors as these have all been studied as matters related to housing insecurity that do not fall directly into the other pillars of housing insecurity. The following is an interdisciplinary review of how these factors affect housing insecurity.  

\subsubsection{\textit{Employment}}

In the United States, the labor market is the result of cumulative individual behaviors including geographical migration and educational investments \hl{(Wiener, 2020)}. The demand for labor is driven by firms, which must consider a wide variety of factors in deciding location \hl{(Partridge and Rickman, 2007)}. In recent decades, the United States labor market has entered a risk regime job market where workers hold a greater share of the risk in an employment system without the perceived promise of security and stability, which has become embedded in American social and political institutions (Lowe, 2018). It is agreed that the Fordist regime that brought unprecedented prosperity in the early 20\textsuperscript{th} century came to an end in the 1970s \hl{(Stockhammer 2008)}. Since this shift, productivity of the average worker has increased 64.6 percent while hourly pay has only increased an average of 17.3 percent between 1979 and 2021 \hl(Economic Policy Institute). Over this same period, U.S. Housing and Urban Development data shows that the median price of a new single-family home increased from \$60,600 (\$232,091 adjusted for inflation) in Q1 of 1979 to \$369,800 in Q1 of 2021) \citep{us_census_bureau_median_1963}. The great recession has had a lasting impact on the housing market within the United States. As the economic recovery did not benefit all households equally, wealth inequality has grown along both racial and ethnic lines \hl{(Lochar, 2014)}. As wages have failed to keep up with the price of housing, the current economic system under this risk regime places those with low incomes in a precarious situation for housing affordability and residential stability. Thus, employment insecurity and income inequality are two pressing issues the United States is facing that have serious impacts on communities. “housing insecurity has risen in relative lockstep with employment insecurity” \citep[48]{desmond_housing_2016-1}. 

Rural communities have been hit hard by economic change, driven by the transition from a production to a consumption based economy \hl{(Pendall et al., 2016)}. what \hl{(Bjerke and Mellander, 2019)} identified an increasing economic divide between urban and rural areas where over several decades rural areas have lost out. During this shift, employment became increasingly scarce for agricultural workers \citep{kropczynski_insights_2012}. Today, manufacturing is responsible for 21 percent of rural non-agricultural earnings \citep{low_rural_2017}. Economic development is therefore a fundamental issue to rural areas. While manufacturing has grown, the majority of counties that experienced manufacturing employment growth between 2001 and 2015 had low levels of growth in terms of total employment \citep{low_rural_2017}. \citet{sherrieb_measuring_2010} identify three key elements connected of economic development: the level of economic resources, the level of equality in resource distribution, level of diversity in economic resources. Economic development alongside demographic change in rural areas has been linked to the quality and condition of local housing infrastructure \citep{barcus_heterogeneity_2011}. Thus, how policies shape economic development has a direct affect on the overall housing insecurity risk of rural communities. Demonstrating the interconnectedness of communities, regional economic development in one area can encourage economic stability of its neighboring regions as well \citep{chen_economic_2018}. \citet{deller_spatial_2016} highlight the importance of economic diversity, a vital aspect of economic development, finding that more diverse economies enhance economic stability. As an insulator against economic instability, employment diversity in rural areas is a key factor that policy makers and scholars should consider as part of a holistic approach to housing insecurity. This may be difficult to achieve for rural areas based on natural amenities, where one industry acts as the lifeblood of the community. 

 

\subsubsection{\textit{Housing, race, and poverty}}
Housing is affected by a variety of social, political, and economic factors. “The ability of residents to access affordable housing, whether renting or buying, is in large part determined by their demographic characteristics, such as income, race, age, and educational attainment” \citep[115]{yadavalli_comprehensive_2020}. While unpredictable events may narrow the disparities, “As a rule, a household’s vulnerability to displacement should be shaped in predictable fashion by those characteristics that define its members’ position in the [social] stratification system” \citep[5]{lee_forced_2020}. Although the federal government took a direct interest in promoting home ownership in 1933, racial discrimination in the housing market was not outlawed until 1968 but enforcement of the law remained difficult until the Fair Housing Act of 1988 (Sharp \& Hall, 2014). For example, the practice of redlining made it difficult for Black Americans to receive mortgages under federal aid programs and creating racial segregation that can still be seen today. At the county level, the probability of living in affordable housing decreases as the white population decreases (Brooks, 2022). In addition to racial segregation, income segregation must be considered for a holistic discussion of housing insecurity. \citet{lichter_rural_2011} found that 40.5 percent of high-poverty places are in high-poverty counties for non-metro areas and the poor and non-poor are becoming increasingly segregated, with higher concentrated poverty among minorities. High concentration of poverty may exacerbate housing condition issues due to a lack of revenue to maintain the necessary services at the household and local government levels. Minorities are also at a disadvantage in income segregation with poor whites being less segregated from their non-poor counterparts \citep{lichter_ruralurban_2021}. As a home is often a household's greatest source of wealth, the disadvantages minorities have in terms of housing are compounded as social and economic inequality are reproduced as these disparities continue \citep{krivo_housing_2004}.  

Rural areas face significant consequences for the historical forces that shape housing today. When discussing rural poverty it must be noted that there is an underlying assumption that the dynamics of poverty are fundamentally different from urban areas \hl{(Valasik, 2018)}. Persistent problems faced by the rural poor include "physical isolation and poor public transportation, inadequate schools, and limit access to medical care and other basic public services while institutional support services are frequently limited or simply unavailable" \citep[?]{lichter_changing_2007}. Part of this is driven by the outflow from rural areas to urban areas. Rural areas have seen a reduction in population, reducing the capabilities of public services to accommodate those in need \hl{(Bjerke and Mellander, 2019)}. As mentioned earlier, there are a variety of reasons why households move. In rural areas, a common reason to move is due to the friction that exists when households are too far removed from labor markets that provide adequate employment and income opportunities \citep{sparks_poverty_2013}. \hl{Valasik (2018)} found that from 2000 to 2012, increases in poverty were larger in rural counties than urban counties with the highest increases in exposure and the rural black population was by far the most disadvantaged over this time period. Rural areas are not as diverse as the United States overall, and many rural minorities are geographically central in regions tied to historical and economic dynamics \hl{(Housing Assistance Council, ?)}. Another demographic group that is significant to rural areas is Hispanics and Latinos, despite the widespread population decline of rural areas \citep{lichter_demographic_2020}. African Americans and Hispanics and Latinos face similar discrimination in the housing market with the benefits of housing are dramatically smaller for these demographics \citep{krivo_housing_2004}. Thus, the pockets of these groups in rural areas should be considered to be at a higher risk of housing insecurity due to the effects of these historical forces. 

\subsubsection{\textit{Housing Type}} 
While owning a home is considered a part of the “American Dream,” many households rent their housing by choice or by necessity. While the many benefits of home ownership portray it as a means to a better life, renting is not inherently bad and may provide better opportunities for households that can afford it, but there are many potentially destabilizing consequences of high-cost renting \citep{drew_believing_2014}. Nationally, median rent in a poor neighborhood is \$298 compared to \$225 in a middle-class neighborhood or \$250 in an affluent neighborhood after regular expenses are deducted despite property values typically being much higher in middle-class or affluent neighborhoods \citep{desmond_poor_2019}. This creates a compounding factor for the previously mentioned disparities in home ownership. Increases in household wealth and secured debt were found to decrease the likelihood of homeowners becoming renters and vice versa \citep{anderson_effect_2021}. Money paid towards a mortgage generates long-term wealth while money paid towards rent generates wealth for the property owner. Renters with high-cost housing are unable to increase household wealth through their means of housing.  In addition to whether one rents or owns their home, the type of home can play a significant role in housing insecurity. Of particular concernr is "unconventional housing' which includes dwellings not considered long-term habitation including RVs/ campers, vans, and boats. These unconventional forms of housing may keep people off the streets, but they are not always a stable mode of housing. For RV and camper living, people who are undocumented or are unable to keep up with legal or maintenance costs for vehicles end up losing their housing \citep{wakin_not_2005}. Mobile homes also carry a unique set of circumstances that may put households at a greater risk of housing insecurity and are found frequently in rural areas. Mobile homes and the land they are situated on can be either owned or rented. It is common in mobile home parks for households to own their home but not the land it is on. Key issues with mobile homes include their financing: typically done through more expensive but easier obtained means than a mortgage such as personal property or chattel loans; mobile homes do not build wealth in the same way as they typically depreciate rather than appreciate; households on rented land have little control over their length of stay; they also tend to have worse construction and higher risks of air pollution and fire than traditional homes \citep{mactavish_wrong_2007}. Those that live in mobile homes or unconventional housing should be a priority for discussing housing in rural areas. 


\subsubsection{\textit{Household income, aid, and Transportation }}
In his first State of the Union address, President Lyndon B. Johnson asked Congress to declare an “unconditional war on poverty… not only to relieve the symptom of poverty but to cure it and, above all, to prevent it.” Since then, a patchwork of programs regulated at the federal, state, and local levels have arisen. A large part of the federal government growth in the late 20th century is from the expansion of social welfare spending \citep{fishback_social_2020}. As the primary mechanism of income distribution is what \citet{berkowitz_gaps_2023} refer to as the “factor payment system” in which those who work and those who own the means of production and one’s relation to this system and the labor market is closely related to one’s poverty risk. To alleviate this poverty risk, social programs which utilize different mechanisms are available to those who qualify. These mechanisms can be divided into categorical and income targeted policy designs, alongside decentralization, where some receive benefits based on “demographically defined, categorical eligibility structures” and others enjoy standardized federal assistance through social insurance with some qualifying for income-based or “means-tested” programs \citep{bruch_poverty_2023}. Households must fall below certain income and asset thresholds to qualify for means tested programs \citep{rank_welfare_2002}. For housing, there is a wide variety of housing policies and programs aimed at low-income individuals. These take the shape of voucher programs by subsidizing privately held property although some recipients live in public housing \citep{kim_housing_2017}. For rural areas, the U.S. Department of Agriculture (USDA) has a variety of programs aimed at improving living conditions in rural areas including direct or guaranteed loans for single or multi-family housing, and infrastructure programs for water, electricity, and telecommunications ((USDA Rural Development Summary of Programs, 2023). Transportation plays a large role in social and economic life. Access to everything from education to healthcare depends on the infrastructure and ability to use available means of transportation. The expense of owning enough vehicles may prove restrictive, especially for households with high housing costs. Rural areas often do not have public transportation, leading residents to depend more on automobiles. An analysis of 2009 National Household Travel Survey data found that 72 percent of households with a yearly income of \$20,000 have access to a household vehicle compared to over 97 percent of households making \$50,000 \citep{blumenberg_automobile_2012}. Automobile ownership can be a crucial factor in avoiding residential instability (Kang, 2019). Households are twice as likely to be auto-deficient (less than 1 car per driver) than zero-vehicle households where a vehicle is not needed \citep{blumenberg_car-deficit_2020}. This is a harrowing statistic in rural areas without public transportation where distances may be too far to walk or ride a bicycle or it may not be safe due to lacking road infrastructure like bicycle lanes and sidewalks.  

\section{\textit{Summary}}
Throughout this chapter, the 4 C's of housing insecurity have been covered. It is important to highlight the interconnected nature of the 4 C's. There is significant overlap between each pillar of housing insecurity. Housing costs, housing type, and housing conditions are necessarily linked to the economic conditions of a household. These economic conditions are linked to the household wage/ aid factors that encapsulate their economic status. One's relation to the poverty level and education plays a significant role in housing accessibility and these factors are intrinsically linked to the context that they grew up in.  For rural areas where public transportation is scarce, one's access to adequate transportation is highly linked to one's economic opportunity. Any discussion on housing insecurity must consider the historical forces affecting modern day race and poverty, and these forces relate to all aspects of life. When taken as a web, this model encompasses the wide ranging socio-economic factors that surround housing insecurity. 
\endinput