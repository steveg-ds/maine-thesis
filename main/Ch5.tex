\chapter{Discussion and Future Directions}	



Housing insecurity affects all aspects of life for the individuals that experience it and it has grave consequences for communities. This is especially true for rural areas where a conundrum of factors over recent decades has reduced the amount of community resources available for combatting housing insecurity and homelessness. These problems include \hl{insert list of rural problems from literature review here}. Unfortunately, our understanding of how these factors affect housing insecurity and homelessness in rural areas is limited due to the urban-centric lens used by researchers and policy-makers. The present analyses examined variables associated with housing insecurity in a sample of rural census tracts based on RUCA designations to group census tracts into risk levels that show similar signs of housing insecurity risk. Several data mining techniques were then applied to analyze how risk levels and variables relate to each other.

\section{\textit{Interpretation of Findings: RQ1}}

\hl{How can risk factors of be used to identify risk levels of housing insecurity while accounting for the variation in rural areas and what do the risk levels say about rural areas?}

K-medoid clustering was used to cluster census tracts with similar values across 8 different sectors of variables, encapsulating different aspects of housing insecurity. To account for the variation in rural areas, a neighbors algorithm was used so that bordering census tracts that make up rural communities could be included in the clustering for each state. There are three benefits to this approach. First, clustering by each state and neighboring census tracts prevents the most vulnerable communities in one state from influencing the risk level of the most vulnerable communities in another state. Second, this state-by-state approach makes this research a tool that policy-makers and researchers can use in their states because everything is based at the state level. Finally and most significantly, a relative approach to measuring housing insecurity can capture the variation in rural areas better than an absolute approach with rigidly defined categories. Whether or not this methodology adequately allows for these differences requires further research to validate. 

 The cluster analysis highlights \hl{?} important observations about rural areas. First is the importance of education, health, and social work employment in rural areas (See Table~\ref{tab:emp}). The overall average for this variable was 10 percent and the cluster medians for each risk level is 9 percent. This is significant because the variables for jobs that rural areas typically depend on like agriculture, construction, and forestry, have notably lower averages and cluster medians. The impact of this is two-fold. First, many of the jobs that fall into education, health, and social work fall into employment in the public sphere, which is greatly affected by the decreases in funding caused by several processes affecting rural areas \hl{(Source)}. Second, it demonstrates the decline of manufacturing and agriculture in rural areas. These industries used to dominate rural areas, but scholars have identified a significant decline in their prevalence over several decades \hl{(Source)}. 
 
 The demographic variable cluster results confirm previous studies on the presence of pockets of minorities in rural areas. The cluster with the lowest white percentage of the population has the highest percentage of Black, Hispanic or Latino, and Other race variables \hl{(Source)}. 20 percent of census tracts in this dataset fall into this category with an average of 6 percent African American population and 12 percent Hispanic and Latino population. Concentrations of African Americans are noteworthy because they are subjected to a variety of historical processes that put them at a higher risk of housing insecurity. Particularly concerning are concentrations of African Americans in the South, where the effects of segregation are still seen today. The presence of Hispanics and Latinos in the same cluster as the lowest white population and highest median black population is interesting because the migration of Hispanics and Latinos to rural areas has been indicated as a potential solution to the well-documented population issues facing rural areas \hl{(Source)}.
 
 Next, the cluster medians indicate that while home ownership is widespread in rural areas, there are notable levels of renters that spend more than 30 percent of their income on housing. The overall average percent of renters with high housing costs is 15 percent compared to only 5 percent for homeowners with a mortgage that have high housing costs. This indicates that renters in rural areas may face similar issues as those in urban areas when it comes to the affordability of rental property. One issue noted not fully accounted for here is the presence of mobile home parks in rural areas. In areas categorized as high-risk for housing costs, an average of 15 percent of renters rent mobile homes while 60 percent rent single-unit homes. An average of 1 percent more renters rent mobile homes in low-risk housing cost census tracts. However, these results do not accommodate some of the extra expenses that come from mobile home renting or ownership such as lot rent and other fees that may come with living in a mobile home park. 
 
 The RME sector shows relatively low levels of transiency among those who have either a high school diploma or did not finish high school. Researchers should be concerned with the average of 8 percent of the population that are stable but do not have a high school education. The value of a high school diploma is well understood, so areas at high risk for RME instability require further attention. An area of concern is that households with low levels of education do not take advantage of public assistance, based on the average, only 1 percent of households with lower levels of education use public assistance compared to the rest of the population. 
 
 One notable observation from the RMP cluster analysis is that the number of households that live below the poverty level is higher than those who are slightly above the poverty level across all three risk levels. It is of great concern for researchers that poverty rates below the poverty level appear to be greater than the rate of households just above the poverty level. The rate of public assistance use is the same for this sector as it was for RME. This indicates that in rural areas, the most vulnerable populations are not taking advantage of available public assistance that could improve their situations. 
 
 The cluster analysis shows 3 points of concern in rural areas for the household factors section. First, for households with no investment income and no other income, all cluster medians are above 30 percent. This indicates that the majority of households in rural areas are not building wealth through means outside of wages received from employment. \hl{importance of non-wage income}. Second, a median of 5 percent of the population receives public assistance across all risk levels. This reflects that rural areas may not be fully taking advantage of assistance that could improve their living conditions. Previous research has documented the tendency for people to not use public assistance for various reasons \hl{(source)}. Third, the median Gini index for each risk level is between 0.42 and 0.44, indicating that income inequality may not be as prevalent of an issue in rural areas as it has been identified in rural areas.

The cluster analysis for housing type shows that single-unit renters and owners are the predominant means of housing in rural areas. One area of concern not accounted for here is the presence of mobile home parks. Research has shown that practices allowed in mobile home parks can put some at a higher risk of housing insecurity \hl{(source)}. For the high-risk housing type census tracts, an average of 13 and 15 percent of owners and renters live in mobile homes. These areas where there are higher rates of renters relative to owners than other sectors warrant further attention because there may be some unidentified factors in the community that contribute to the lower amounts of home ownership. 

\section{\textit{Interpretation of Findings: RQ2}}

\hl{When measuring housing insecurity across different dimensions, how often do the same features arise?}

One important question when using the 4 C's model is how sector risk levels relate to each other. An area at an overall high risk of housing insecurity would at least have a high-risk level across more than one sector. Tables~\ref{tab:high_risk_ass},~\ref{tab:low_risk_ass},~\ref{tab:low_high_risk},~\ref{tab:high_low_risk} show the frequency of the most interesting relationships: high-risk-to-high-risk, low-risk-to-low-risk, low-risk-to-high-risk, and high-risk-to-high-risk. The most surprising result from this analysis is the similar levels of presence between all of these relationships. These indicate that areas with a high risk of housing insecurity in one sector may face a high risk of insecurity in another area, or it may be at a low risk of insecurity in another sector. The low support and fairly low confidence levels demonstrated across association rules indicate few areas that exhibit a risk of housing insecurity across multiple sectors. This is reflected by Figure~\ref{fig:regional_map} where the map of housing insecurity shows some pockets of red, but vast amounts of green and yellow indicated low and medium risk of housing insecurity across sectors for most rural areas.The lack of a significant number of rules with more than one element on the left-hand side provides further evidence for this hypothesis. This raises further questions related to the clustering of housing insecurity factors among rural areas as is often seen in urban areas. 

\section{\textit{Interpretation of Findings: RQ3}}

\hl{Are there spatial relations between the different dimensions of housing insecurity? }

Global Moran's I was calculated for each state and nationwide. Additionally, local Moran's I was used on the risk levels to determine their spatial autocorrelation at the census tract level. The first notable observation from the global Moran's I result is the strength of the key demographic variables. African Americans, Hispanics and Latinos, White, american indian and native Alaskan, and Other all have spatial autocorrelations greater than 0.5 nationwide. This offers strong support for previous research that has found pockets of minorities in rural areas \hl{(source)}.  While the percentage of rural economies that manufacturing and agriculture, forestry, fishing, hunting, and mining make up has declined in recent years, there is a significant spatial autocorrelation to both of these variables. This is reflective of the amenities-based nature of these amenities and further enforces the role of single industry-based economies in rural areas. Economic diversity is generally seen as a good thing, so the high spatial autocorrelation of these two industries is concerning for the overall economic diversity of rural areas. Another concern is the relatively high spatial autocorrelation of households that did not move, but do not have a high school diploma. This could be reflective of areas where schools have suffered due to the declines facing rural areas \hl{(Source)}. 



\section{\textit{Interpretation of Findings: RQ4}}
\hl{To what extent can this model of housing insecurity be used to predict risk levels across housing insecurity factors?}

In total, \hl{?} multinomial logistic regression models were used to predict the risk levels of each sector for each state and nationally. The results are very low probabilities of the model predicting the correct cluster, and very low accuracy. Most notable for the national models shown in Chapter 4's confusion matrices is their tendency to over-classify census tracts as low-risk for every sector. One explanation for this is class imbalance. Figure~\ref{fig:cluster_dis} shows that there are class imbalances mostly due to high levels of lower risk levels. This could have led the models into overclassifying census tracts as low-risk when they are not. The lack of accuracy of these models echoes what the literature has said: "rural" is not a blanket term but rather, it encompasses a wide-ranging and varying group of areas. The most common thing between rural areas may simply be that they are not urban. 

\section{\textit{Limitations}}

There are three significant limitations to this work. First,Due to the urban-centric lens towards housing insecurity, there is little previous research to compare to this study. Gleason et al. 2021 applied similar spatial techniques to census tracts in Maine and found that poverty, unemployment, and high housing costs are common in rural and urban areas of Maine but found these results to be inaccurate in a later study (Gleason et al 2022?). The second limitation is due the lower rate of ACS sampling in rural areas, the accuracy of the data is limited in how well it represents the real-world. While the estimates are “likely reasonable approximations of the populations they represent”, small area estimates like census tracts used here have issues with attribute uncertainty (Spielman 2014, ?). Despite this, it is currently the most detailed source of data available for rural areas. The final limitation is that as the risk-level assignment system is relative, it cannot be used to make definitive claims about the housing insecurity risk of an area.  

\section{\textit{Future Research}}

Future research should use this study as a starting point for giving housing insecurity and homelessness adequate attention. The most important direction is to identify community level risk factors unique to rural areas. Further studies should also use a wider range of data sources to capture sectors with few available variables such as housing conditions. From our spatial analysis there are several observations worthy of further investigation. Subsequent investigations should focus on examining rural housing insecurity at a localized level. This will enable the refinement and enhancement of the existing model, providing more precise insights into the unique challenges faced by rural communities.	Future endeavors should prioritize a closer examination of regions exhibiting unexpected high-risk-to-low-risk and low-risk-to-high-risk relationships, as identified through association rules. Understanding the underlying factors contributing to these unexpected relationships is essential for targeted interventions and policy recommendations. There is a need for in-depth research to discern how levels and trends in income inequality differ between urban and rural areas, shedding light on the specific socio-economic dynamics impacting housing insecurity in each setting. Future research should scrutinize the distinctions in poverty and housing cost dynamics between rural and urban areas, aiming to gain a deeper understanding of the factors at play in each context. The states highlighted in the Moran's Outlier section warrant attention because they exhibit notably higher levels of spatial clustering of risk factors than other states. People in these states may be at a higher risk of housing insecurity relative to other states. By addressing these research gaps, researchers can better inform evidence-based policies and interventions that mitigate housing insecurity and advance the well-being of rural populations. As we strive to enhance housing security and social equity in both rural and urban landscapes, interdisciplinary collaboration and persistent research efforts will remain pivotal in driving meaningful societal change. 


\endinput